\documentclass[12pt]{report}

\usepackage{amssymb,amsmath}
\usepackage[utf8]{inputenc}
\usepackage[spanish,mexico]{babel}

\usepackage{iwona}

\voffset=-3cm
\hoffset=-3 cm
\textwidth=19 cm
\textheight=30 cm

\begin{document}

\begin{center}
\textbf{\LARGE {GEOMETRÍA MODERNA I}}
\end{center}

\begin{center}
\textbf{{\large 2019-1 (23 noviembre 2018)}}
\end{center}

\begin{center}
\textbf{{\large EXAMEN PARCIAL 05}}
\end{center}

\vspace{1cm}

{\bf INSTRUCCIONES:} Justificar y argumentar todos los resultados que se realicen. Resolver únicamente cinco ejercicios, de entregar más de cinco ejercicios se anulará el ejercicio de mayor puntaje.

\vspace{1cm}

\begin{enumerate}

\item Demostrar que si $l$ y $m$ son dos rectas distintas en el plano, $\{A,C,E\}\subseteq l$ distintos, $\{B,D,E\}\subseteq m$ distintos y $\overline{AB}\cap\overline{DE}=\{P\}$, $\overline{BC}\cap\overline{EF}=\{Q\}$, $\overline{CD}\cap\overline{AF}=\{R\}$ entonces $\{P,Q,R\}$ es un conjunto de puntos colineales.

\item Demostrar que si $\zeta(O,r)$ es la circunferencia que inscribe al cuadrado $\square ABCD$ (con los vértices ordenados levógira o dextrógiramente sobre la circunferencia) entonces para cualquier \break  $P\in \zeta(O,r)\setminus \{A,B,C,D\}$ se tiene que $P(\overline{PA},\overline{PC};\overline{PB},\overline{PD})$.

\item Demostrar que cada unos de los triángulos formados por tres de los cuatro lados de un cuadrilátero completo está en perspectiva con el triángulo diagonal del cuadrilátero.

\item Construir un cuadrángulo completo que tenga un triángulo dado como triángulo diagonal.

\item Sea $\square ABCD$ un cuadrángulo.

Demostrar que existe $\zeta(O,r)$ tal que $\{A,B,C,D\}\subseteq \zeta(O,r)$ si y solo si $AD\cdot BC + AB \cdot CD = AC \cdot BD$.

\item Sean $\triangle ABC$, $\zeta(O,r)$ la circunferencia que lo inscribe y $P\in \zeta(O,r)\setminus \{A,B,C\}$. Demostrar que si $l_{XY}$ es la recta ortogonal a $\overline{XY}$ incidente en $P$ con $\{X,Y\} \subseteq \{A,B,C\}$ y $X\neq Y$ y $l_{AB}\cap \overline{AB}=\{Q\}$, $l_{BC}\cap \overline{BC}=\{R\}$, $l_{AC}\cap \overline{AC}=\{S\}$ entonces $\{Q,R,S\}$ es un conjunto de puntos colineales.

\end{enumerate}




\end{document}
