\documentclass[12pt]{report}

\usepackage{amssymb,amsmath}
\usepackage[utf8]{inputenc}
\usepackage[spanish,mexico]{babel}

\usepackage{iwona}

\voffset=-3cm
\hoffset=-3 cm
\textwidth=19 cm
\textheight=30 cm

\begin{document}

\begin{center}
\textbf{\LARGE {GEOMETRÍA MODERNA I}}
\end{center}

\begin{center}
\textbf{{\large 2019-1 (7 diciembre 2018)}}
\end{center}

\begin{center}
\textbf{{\large EXAMEN FINAL}}
\end{center}

\vspace{1cm}

{\bf INSTRUCCIONES:} Justificar y argumentar todos los resultados que se realicen. Resolver los cinco ejercicios.

\vspace{1cm}

\begin{enumerate}

\item Sea $l$ una recta y $L$ un punto tal que $L\notin l$. Construir la recta incidente en $L$ que es paralela a $l$.

\item Demostrar que si $\triangle ABC$ se cumple que $|\angle ABC|=\perp$, $L\in \overline{AC}$ con $|AL|=|LC|$, $M$ el punto en que la bisectriz interna de $|\angle ABC|$ interseca a $\overline{AC}$, $h_B$ la ortogonal a $\overline{AC}$ por $B$ y $N= h_B \cap \overline{AC}$ entonces $|\angle NBM|=|\angle MBL|$.

\item Sean $A,O, H$ tres puntos no colineales. Construir un triángulo que tenga a $A$ como uno de sus vértices, a $H$ como su ortocentro y a $O$ como su circuncentro.

\item Sean $\triangle ABC$, $P \in \overline{BC}$, $Q \in \overline{CA}$ y $R \in \overline{AB}$ tal que $\overline{AP}\cap \overline{BQ}\cap \overline{CR}\neq \emptyset$ . Demostrar que 

\begin{itemize}

\item si $\overline{QR}\cap \overline {BC}= \{P'\}$, $\overline{RP}\cap \overline {CA}= \{Q'\}$, $\overline{PQ}\cap \overline {AB}= \{R'\}$ entonces $P',Q', R'$ son colineales.

\item $\overline{AP}, \overline{BQ'}$ y $\overline{CR'}$ son concurrentes.

\end{itemize}

\item Demstrar que cada uno de los triángulos formados por tres de los cuatro lados de un cuadrilátero completo está en perspectiva con el triángulo diagonal del cuadrilátero.

\end{enumerate}




\end{document}
