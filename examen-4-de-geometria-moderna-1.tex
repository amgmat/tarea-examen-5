\documentclass[12pts]{report}


\usepackage{amssymb}
\usepackage{amsmath}
\usepackage{amscd}
\usepackage{amsthm}
\usepackage[utf8]{inputenc}
\usepackage[spanish,mexico]{babel}
\usepackage{enumerate}
\usepackage[usenames]{color}


\usepackage{pgf,tikz}
\usetikzlibrary{arrows}

\usepackage[colorlinks=true, linkcolor=blue, urlcolor=red,
citecolor=green]{hyperref}

\voffset=-2cm
\hoffset=-2cm
\textwidth = 18cm
\textheight= 23 cm

\usepackage{iwona}
\usepackage{fancyhdr}
\pagestyle{fancy}
\fancyhf{}
\fancyhead[RE,LO]{\bfseries{Geometría Moderna 1}}
\fancyhead[LE,RO]{\bfseries{2019-1}}
\fancyfoot[RE,RO]{\bfseries{Noviembre 2018}}
\fancyfoot[LE,LO]{\bfseries{Examen 04}}

\newcommand{\R}{\mathbb R}
\newcommand{\Q}{\mathbb Q}
\newcommand{\E}{\mathbb E}
\newcommand{\s}{\mathbb S}
\newcommand{\C}{\mathbb C}
\newcommand{\F}{\mathbb F}
\newcommand{\T}{\mathbb T}
\newcommand{\p}{\mathbb P}
\newcommand{\I}{\mathbb I}
\newcommand{\A}{\mathbb A}
\newcommand{\h}{\mathbb H}

\begin{document}
\begin{center}
\textbf{{\large EXAMEN PARCIAL 04}}
\end{center}

{\bf INSTRUCCIONES:} Justificar y argumentar todos los resultados que se realicen. Resolver únicamente cuatro ejercicios, de entregar más de cuatro ejercicios se anulará el ejercicio de mayor puntaje.


\begin{enumerate}

\item Sea$\triangle ABC$. Demostrar que si $P\in\overline{AB}$, $Q\in\overline{AC}$, $\overline{PQ}$ es paralela a $\overline{BC}$ y $\overline{BQ} \cap \overline{CP}=\{O\}$ entonces $\overline{AO}$ es una mediana del $\triangle ABC$.

\item Sea \{A,B,C,D\} cuatro puntos que por tercias no están en la misma recta. Demostrar que si $\overline{AB}$, $\overline{BC}$, $\overline{CD}$ y $\overline{DA}$ son cuatro rectas no concurrentes y son cortadas por una recta $l$ con la propiedad de que $\{A,B,C,D\}\cap l=\emptyset$ en los puntos $\overline{AB}\cap l=\{P\}$, $\overline{BC}\cap l=\{Q\}$, $\overline{CD}\cap l=\{R\}$ y $\overline{DA}\cap l=\{S\}$ entonces
$$\frac{AP}{PB}.\frac{BQ}{QC}.\frac{CR}{RD}.\frac{DS}{SA}=1$$

\item Demostrar que si $\mathcal{C}(I,r)$ es la circunferencia inscrita del $\triangle ABC$ y $\mathcal{C}\cap\overline{BC}=\{P\}$, $\mathcal{C}\cap\overline{CA}=\{Q\}$ y $\mathcal{C}\cap\overline{AB}=\{R\}$ entonces $\overline{AP}$, $\overline{BQ}$ y $\overline{CR}$ son concurrentes.

\item Sea $\mathcal{C}$ una circunferencia y $\{,B,C,D,E,F\}\subset \mathcal{C}$ ordenados (levogiramente o dextrogiramente). Demostrar que la interseccion de los lados opuestos de henagono inscrito son tres puntos colineales.\textbf{Sugerencia}: Considerar a $\overline{AB}\cap\overline{CD}=\{P\}$, $\overline{CD}\cap\overline{EF}=\{Q\}$ y $\overline{EF}\cap\overline{AB}=\{R\}$.

\item Usando unicamente regla, encontrar la recta que une a un punto $P$ del plano con la interseccion de dos rectas dadas sin usar el punto de interseccion.
\end{enumerate}
\end{document}