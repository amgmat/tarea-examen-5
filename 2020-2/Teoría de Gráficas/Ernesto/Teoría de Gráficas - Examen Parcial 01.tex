\documentclass[10pt]{report}

\usepackage{amssymb,amsmath}
\usepackage[utf8]{inputenc}
\usepackage[spanish,mexico]{babel}

\usepackage{iwona}

\voffset=-3cm
\hoffset=-4cm
\textwidth=20 cm
\textheight=26 cm

\newcommand{\R}{\mathbb R}
\newcommand{\Q}{\mathbb Q}
\newcommand{\E}{\mathbb E}
\newcommand{\s}{\mathbb S}
\newcommand{\C}{\mathbb C}
\newcommand{\F}{\mathbb F}
\newcommand{\T}{\mathbb T}
\newcommand{\p}{\mathbb P}
\newcommand{\I}{\mathbb I}
\newcommand{\A}{\mathbb A}

\begin{document}

\begin{center}
\textbf{\LARGE {TEORÍA DE GRÁFICAS}}
\end{center}

\begin{center}
\textbf{{\large 2020-2 (28 febrero 2020)}}
\end{center}

\begin{center}
\textbf{{\large EXAMEN PARCIAL 01}}
\end{center}

{\bf INSTRUCCIONES:} 
\begin{itemize}
\item Justificar y argumentar todos los resultados que se realicen.
\item Se cuentan con 60 minutos para resolver dos de los primeros tres ejercicios.
\item Los ejercicios restantes se deben entregar resueltos el día 02 de marzo de 2020 a las 9:00 horas en el salón P-118.
\end{itemize}

\begin{center}
\rule[0mm]{20cm}{0.2mm}
\end{center}

\begin{enumerate}

\item Sean $G$ y $H$ gráficas tal que $G \cong H$. Demostrar que si $\varphi$ es un isomorfismo entre $G$ y $H$ entonces para cualquier $x \in V(G)$ se tiene que $d_G(x) = d_H(\varphi(x))$.

\item Describir, por medio de una gráfica, a un grupo de cinco amigos en los que cualesquiera dos de ellos tienen exactamente un amigo en común. ¿Es posible hacer lo mismo con un grupo de cuatro amigos?

\item Demostrar que si $G$ es disconexa entonces $\overline{G}$ es conexa. 

\item
\begin{itemize}
\item Demostrar que en cualquier gráfica $G$, si ${|V(G)| -1 \choose 2} < |A(G)|$ entonces $G$ es conexa.
\item Encontrar, para cada $n \in \mathbb{N} \setminus \{0,1\}$, una gráfica disconexa $G$ de orden $n$ tal que ${|V(G)| -1 \choose 2} = |A(G)|$.
\end{itemize}

\item Demostrar que, para cualquier $n \in \mathbb{N} \setminus \{0\}$, $Q_n \cong B_n$.

\end{enumerate}


\vspace{2cm}


\begin{center}
\textbf{\LARGE {TEORÍA DE GRÁFICAS}}
\end{center}

\begin{center}
\textbf{{\large 2020-2 (28 febrero 2020)}}
\end{center}

\begin{center}
\textbf{{\large EXAMEN PARCIAL 01}}
\end{center}

{\bf INSTRUCCIONES:} 
\begin{itemize}
\item Justificar y argumentar todos los resultados que se realicen.
\item Se cuentan con 60 minutos para resolver dos de los primeros tres ejercicios.
\item Los ejercicios restantes se deben entregar resueltos el día 02 de marzo de 2020 a las 9:00 horas en el salón P-118.
\end{itemize}

\begin{center}
\rule[0mm]{20cm}{0.2mm}
\end{center}

\begin{enumerate}

\item Sean $G$ y $H$ gráficas tal que $G \cong H$. Demostrar que si $\varphi$ es un isomorfismo entre $G$ y $H$ entonces para cualquier $x \in V(G)$ se tiene que $d_G(x) = d_H(\varphi(x))$.

\item Describir, por medio de una gráfica, a un grupo de cinco amigos en los que cualesquiera dos de ellos tienen exactamente un amigo en común. ¿Es posible hacer lo mismo con un grupo de cuatro amigos?

\item Demostrar que si $G$ es disconexa entonces $\overline{G}$ es conexa. 

\item
\begin{itemize}
\item Demostrar que en cualquier gráfica $G$, si ${|V(G)| -1 \choose 2} < |A(G)|$ entonces $G$ es conexa.
\item Encontrar, para cada $n \in \mathbb{N} \setminus \{0,1\}$, una gráfica disconexa $G$ de orden $n$ tal que ${|V(G)| -1 \choose 2} = |A(G)|$.
\end{itemize}

\item Demostrar que, para cualquier $n \in \mathbb{N} \setminus \{0\}$, $Q_n \cong B_n$.

\end{enumerate}


\end{document}