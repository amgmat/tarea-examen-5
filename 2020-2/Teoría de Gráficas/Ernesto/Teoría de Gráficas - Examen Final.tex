\documentclass[11pt]{report}

\usepackage{amssymb,amsmath}
\usepackage[utf8]{inputenc}
\usepackage[spanish,mexico]{babel}

\usepackage{iwona}

\voffset=-3cm
\hoffset=-2.5cm
\textwidth=18 cm
\textheight=24 cm

\newcommand{\R}{\mathbb R}
\newcommand{\Z}{\mathbb Z}
\newcommand{\N}{\mathbb N}
\newcommand{\Q}{\mathbb Q}
\newcommand{\E}{\mathbb E}
\newcommand{\s}{\mathbb S}
\newcommand{\C}{\mathbb C}
\newcommand{\F}{\mathbb F}
\newcommand{\T}{\mathbb T}
\newcommand{\p}{\mathbb P}
\newcommand{\I}{\mathbb I}
\newcommand{\A}{\mathbb A}

\begin{document}
\pagestyle{empty}

\begin{center}
\textbf{\LARGE {TEORÍA DE GRÁFICAS}}
\end{center}

\begin{center}
\textbf{{\large 2020-2 (17 junio 2020)}}
\end{center}

\begin{center}
\textbf{{\large EXAMEN FINAL}}
\end{center}

{\bf INSTRUCCIONES:}
\begin{itemize}
\item Justificar y argumentar todos los resultados que se realicen.

\item Resolver y enviar por correo electrónico, en formato \textit{PDF}, a ambos profesores los siguientes ejercicios resueltos.

\item La fecha límite de envío del \textbf{Examen Final} es el
\begin{center}
\textbf{Miércoles 17 de junio de 2020 a las 16:00 horas}
\end{center}
No se considerará a revisión cualquier archivo que se envíe como \textbf{Examen Final} después de esta fecha y horario.

\end{itemize}

\begin{center}
\rule[0mm]{20cm}{0.2mm}
\end{center}

\begin{enumerate}
\item Sea $n \in \N \setminus \{0\}$, considerar las siguientes gráficas:
\begin{itemize}
\item $Q_n = (V(Q_n), A(Q_n))$ donde $X=\{0,1\}$ y
\begin{itemize}
\item $V(Q_n) = X^n $
\item $\{x,y\} \in A(Q_n) \Leftrightarrow$ $x$ difiere de $y$ en exactamente una coordenada. 
\end{itemize}

\item $B_n = (V(B_n), A(B_n))$ donde $X=\{m \in \N \setminus \{0\} \; | \; 1 \leq m \leq n\}$ y
\begin{itemize}
\item $V(B_n) = \wp(X) $
\item $\{x,y\} \in A(B_n) \Leftrightarrow$ $|x \triangle y|=1$.\footnote{$x \triangle y = (x \setminus y) \cup (y \setminus x)$}
\end{itemize}
\end{itemize}

\begin{enumerate}
\item Demostrar que para cualquier $n \in \N \setminus \{0\}$, $Q_n \cong B_n$.
\item Determinar el orden y el tamaño de $Q_n$ para toda $n \in \N \setminus \{0\}$.
\item Demostrar que para toda $n \in \N \setminus \{0\}$ se tiene que $B_n$ es bipartita.
\end{enumerate}

\item Demostrar que toda gráfica autocomplementaria con $4k+1$ vértices tiene un vértice de grado $2k$.

\item Demostrar que si $v \in V(G)$ es vértice de corte en $G$ entonces $v$ no es vértice de corte en $\overline{G}$.

\item Demostrar que si $G$ es un árbol entonces tiene un centro o dos centros que son adyacentes\footnote{Sea $G$ una gráfica. Definimos, para cualquier $v \in V(G)$, $\mathbb{A}_v = \{ k \in \mathbb N \; | \; k = d_G(v,x) \text{ con } x \in V(G)\}$ y sea $a_v = \max(\mathbb A_v)$.

$v \in V(G)$ es un \textbf{centro de $G$} si y solamente si $v$ es un vértice que satisface con ser el vértice de $G$ que representa al $\min(\{ a_v \in \mathbb N \; | \; v \in V(G) \})$}.

\item Demostrar que si $G$ es una gráfica $3$-regular entonces $\kappa(G) = \lambda(G)$.

\end{enumerate}

\end{document}
