\documentclass[10pt]{report}

\usepackage{amssymb}
\usepackage{amsmath}
\usepackage{amscd}
\usepackage{amsthm}
\usepackage[utf8]{inputenc}
\usepackage[spanish,mexico]{babel}
\usepackage{enumerate}
\usepackage[usenames]{color}
\numberwithin{section}{chapter}

\usepackage{pgf,tikz}
\usetikzlibrary{arrows}

\usepackage{multicol}

\usepackage{graphicx}
\usepackage{subfigure}

\usetikzlibrary{knots,hobby,decorations.pathreplacing,shapes.geometric,calc}
\tikzset{knot diagram/every strand/.append style={ultra thick,red}, show curve controls/.style={postaction=decorate, decoration={show path construction, curveto code={\draw [blue, dashed](\tikzinputsegmentfirst) -- (\tikzinputsegmentsupporta) node [at end, draw, solid, red, inner sep=2pt]{}; \draw [blue, dashed] (\tikzinputsegmentsupportb) -- (\tikzinputsegmentlast) node [at start, draw, solid, red, inner sep=2pt]{} node [at end, fill, blue, ellipse, inner sep=2pt]{};}}}, show curve endpoints/.style={ postaction=decorate, decoration={show path construction, curveto code={\node [fill, blue, ellipse, inner sep=2pt] at (\tikzinputsegmentlast) {};}}}}

\usepackage[colorlinks=true, linkcolor=blue, urlcolor=red,
citecolor=green]{hyperref}

\voffset=-2cm
\hoffset=-3cm
\textwidth = 18cm
\textheight= 23 cm

\usepackage{iwona}
\usepackage{fancyhdr}
\pagestyle{fancy}
\fancyhf{}
\fancyhead[RE,LO]{\bfseries{Teoría de Gráficas}}
\fancyhead[LE,RO]{\bfseries{2020-2}}
\fancyfoot[RE,RO]{\bfseries{Abril 2020}}
\fancyfoot[LE,LO]{\bfseries{Evaluación Parcial 03}}

\newcommand{\N}{\mathbb N}
\newcommand{\Q}{\mathbb Q}
\newcommand{\E}{\mathbb E}
\newcommand{\s}{\mathbb S}
\newcommand{\C}{\mathbb C}
\newcommand{\F}{\mathbb F}
\newcommand{\T}{\mathbb T}
\newcommand{\p}{\mathbb P}
\newcommand{\I}{\mathbb I}
\newcommand{\A}{\mathbb A}


\begin{document}
\begin{center}
\textcolor{blue}{\textbf{\large Guía de ejercicios para la Evaluación Parcial 03}}\\
\vspace{0.5 cm}
\textcolor{red}{\textbf{\large FECHA DE EVALUACIÓN PARCIAL 03 \\ JUEVES 30-ABRIL-2020\\ 12:00 HORAS}}
\end{center}

\textbf{Instrucciones}: 
\begin{itemize}
\item Desde el momento de la publicación de esta lista y hasta el \textbf{Martes 28 de abril de 2020 a las 12:00 horas} recibiremos via correo electrónico cualquier duda que tengan al respecto y, de ser necesario, pueden agendar una cita para comunicarnos via {\it Skype} con alguno de los profesores, para aclarar cualquier duda que surja sobre esta tarea o sobre la teoría que comprende la evaluación.

\item \textbf{Derecho a examen}:
\begin{itemize}
\item Deberán enviar via correo electrónico, a ambos profesores del curso, \textbf{dos de los ejercicios de esta guía} a libre elección.

\item Los ejercicios se revisarán y se enviarán al autor de los mismos ya revisados y con comentarios. De no estar correctos, deberán corregirlos para obtener el visto bueno de sus resultados.

\item Solamente al tener el visto bueno de todos sus ejercicios tendrán derecho a presentar el tercer examen parcial. La fecha límite para enviar a revisión los ejercicios es el \textbf{Martes 28 de abril de 2020 a las 12:00 horas}.
\end{itemize}

\item \textbf{Examen parcial}:
\begin{itemize}
\item El tercer examen parcial se publicará en la página del curso el \textbf{Miércoles 29 de abril de 2020 a las 12:00 horas}, deberán resolverlo y enviar sus resultados via correo electrónico a ambos profesores a más tardar el \textbf{Jueves 30 de abril de 2020 a las 12:00 horas}.

\item De no haber recibido un correo electrónico por parte de los profesores argumentando que se ha obtenido el \textbf{Derecho a Examen}, no se considerará a revisión cualquier archivo que se envíe como \textbf{Examen Parcial 03}.
\end{itemize}
\end{itemize}

\vspace{1cm}


\begin{center}
\textcolor{blue}{\textbf{\large LISTA DE EJERCICIOS}}
\end{center}


\begin{enumerate}
\item Demostrar que si $G$ es una gráfica en la que para todo $v \in V(G)$ se tiene que $ 2 \leq d_G(v)$ entonces $G$ contiene un ciclo.

%EXAMEN
%\item Demostrar que si $G$ es una gráfica en la que $|V(G)| \leq |A(G)|$ entonces $G$ contiene un ciclo.

\item Demostrar que si $G$ es un árbol con exactamente dos hojas entonces $G$ es una trayectoria.

%EXAMEN
%\item Demostrar que si $G$ es un árbol tal que $k \leq \Delta(G)$ entonces $G$ tiene al menos $k$ hojas.

\item Demostrar que las siguientes proposiciones son equivalentes:
\begin{itemize}
\item $G$ es una gráfica conexa con $|V(G)|-1 = |A(G)|$.
\item $G$ es un bosque con $|V(G)|-1 = |A(G)|$.
\item $G$ es un árbol.
\end{itemize}

%EXAMEN
%\item Demostrar que $G$ es un bosque si y solamente si $|A(G)| = |V(G)|-cc(G)$.

\item Demostrar que si $G$ es un árbol entonces tiene un centro o dos centros que son adyacentes\footnote{Sea $G$ una gráfica. Definimos, para cualquier $v \in V(G)$, $\mathbb{A}_v = \{ k \in \mathbb N \; | \; k = d_G(v,x) \text{ con } x \in V(G)\}$ y sea $a_v = \max(\mathbb A_v)$.

$v \in V(G)$ es un \textbf{centro de $G$} si y solamente si $v$ es un vértice que satisface con ser el vértice de $G$ que representa al $\min(\{ a_v \in \mathbb N \; | \; v \in V(G) \})$}.

%EXAMEN
%\item Demostrar que una gráfica es conexa si y solamente si tiene un árbol generador.

%EXAMEN
%\item Demostrar que todo árbol es una gráfica bipartita.

\end{enumerate}

\end{document}
