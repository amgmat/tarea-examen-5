\documentclass[12pts]{report}


\usepackage{amssymb}
\usepackage{amsmath}
\usepackage{amscd}
\usepackage{amsthm}
\usepackage[utf8]{inputenc}
\usepackage[spanish,mexico]{babel}
\usepackage{enumerate}
\usepackage[usenames]{color}


\usepackage{pgf,tikz}
\usetikzlibrary{arrows}

\usepackage[colorlinks=true, linkcolor=blue, urlcolor=red,
citecolor=green]{hyperref}

\voffset=-2cm
\hoffset=-2cm
\textwidth = 18cm
\textheight= 23 cm

\usepackage{iwona}
\usepackage{fancyhdr}
\pagestyle{fancy}
\fancyhf{}
\fancyhead[RE,LO]{\bfseries{Teoría de Gráficas}}
\fancyhead[LE,RO]{\bfseries{2020-2}}
\fancyfoot[RE,RO]{\bfseries{Febrero 2020}}
\fancyfoot[LE,LO]{\bfseries{Tarea 01}}

\newcommand{\R}{\mathbb R}
\newcommand{\Q}{\mathbb Q}
\newcommand{\E}{\mathbb E}
\newcommand{\s}{\mathbb S}
\newcommand{\C}{\mathbb C}
\newcommand{\F}{\mathbb F}
\newcommand{\T}{\mathbb T}
\newcommand{\p}{\mathbb P}
\newcommand{\I}{\mathbb I}
\newcommand{\A}{\mathbb A}
\newcommand{\h}{\mathbb H}

\begin{document}
\begin{center}
\textcolor{blue}{\textbf{\large Ejercicios para al Evaluación Parcial 01}}\\
\vspace{0.5 cm}
\textcolor{red}{\textbf{\large FECHA DE ENTREGA \\ VIERNES 28- FEBRERO-2020\\ De 9:00 a 10:00 HORAS - Salón P-118}}
\end{center}

\textbf{Instrucciones}: Resolver

\vspace{1cm}

\begin{enumerate}
\item Sea $G$ una gráfica de orden $n$ y de tamaño $m$.
\begin{enumerate}
\item Demostrar que $m \leq {n \choose 2}$ y determinar cuando se da la igualdad.
\item Demostrar que $m \leq \frac{n^2}{4}$
\end{enumerate}
\item Demostrar que la relación de isomorfismos $(\cong)$ entre gráficas es de equivalencia, es decir:\\
Sean $G$,$H$ y $I$ gráficas
\begin{itemize}
\item Demostrar que $G \cong G$
\item Demostrar que si $G\cong H$ entonces $H\cong G$
\item Demostrar que si $G\cong H$ y $H\cong I$ entonces $G\cong I$.
\end{itemize}
\item Demostrar que una gráfica completa de orden $n$ es única.
\item Sean $G$ y $H$ gráficas tal que $H\cong G$. Si $\varphi$ es el isomorfismo entre $G$ y $H$ entonces para cualquier $x\in G$ se tiene que $d_G(x) = d_H(\varphi(x))$
\item Sea $G[X,Y]$ una gráfica bipartita de orden $n$ y de tamaño $m$, donde $|X|= r$, $|Y|=s$. Demostrar que
$m\leq rs$.

\item Sea $G$ una gráfica. Demostrar que $C$ es un ciclo de $G$ cuya gráfica que induce es bipartita $\Leftrightarrow $ su longitud es par.

\item Demostrar que si $G$ es un gráfica no conexa entonces $\overline{G}$ es conexa. ¿Qué sucede con el regreso?.

\item Sea $G$ una gráfica conexa. Demostrar que para toda $H$ gráfica tal que $G\cong H$ entonces $H$ es conexa.

\item Demostrar la siguiente proposición o dar un contraejemplo en caso de que sea falsa. 
Sea $G$ una gráfica $k$-regular. Si $K$ es impar entonces $G$ tiene un número impar de vértices.

\item Sea $G[X,Y]$ una gráfica bipartita. Demostrar que:
\begin{itemize}
\item $\Sigma_{v\in X} d(v) = \Sigma_{v\in Y} d(v)$.
\item Si $G$ es k-regular con $k>1$ entonces $|X|=|Y|$.
\end{itemize}
\end{enumerate}


\end{document}
