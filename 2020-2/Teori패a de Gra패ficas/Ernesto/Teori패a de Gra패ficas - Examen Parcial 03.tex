\documentclass[12pt]{report}

\usepackage{amssymb,amsmath}
\usepackage[utf8]{inputenc}
\usepackage[spanish,mexico]{babel}

\usepackage{iwona}

\voffset=-3cm
\hoffset=-3cm
\textwidth=19 cm
\textheight=26 cm

\newcommand{\R}{\mathbb R}
\newcommand{\Q}{\mathbb Q}
\newcommand{\E}{\mathbb E}
\newcommand{\s}{\mathbb S}
\newcommand{\C}{\mathbb C}
\newcommand{\F}{\mathbb F}
\newcommand{\T}{\mathbb T}
\newcommand{\p}{\mathbb P}
\newcommand{\I}{\mathbb I}
\newcommand{\A}{\mathbb A}

\begin{document}

\begin{center}
\textbf{\LARGE {TEORÍA DE GRÁFICAS}}
\end{center}

\begin{center}
\textbf{{\large 2020-2 (30 abril 2020)}}
\end{center}

\begin{center}
\textbf{{\large EXAMEN PARCIAL 03}}
\end{center}

{\bf INSTRUCCIONES:}
\begin{itemize}
\item Justificar y argumentar todos los resultados que se realicen.

\item De no haber recibido un correo electrónico por parte de los profesores argumentando que se ha obtenido el \textbf{Derecho a Examen}, no se considerará a revisión cualquier archivo que se envíe como \textbf{Examen Parcial 03}.

\item Resolver y enviar por correo electrónico, a ambos profesores, \textbf{cinco} los siguientes ejercicios a libre elección. De entregar mas de cinco ejercicios, se evaluarán todos los ejercicios pero no se considerará para el promedio el ejercicio de mayor puntaje.

\item La fecha límite de envío del \textbf{Examen Parcial 03} es el
\begin{center}
\textbf{Jueves 30 de abril de 2020 a las 12:00 horas}
\end{center}
No se considerará a revisión cualquier archivo que se envíe como \textbf{Examen Parcial 03} después de esta fecha y horario.

\end{itemize}

\begin{center}
\rule[0mm]{20cm}{0.2mm}
\end{center}

\begin{enumerate}
%\item Demostrar que si $G$ es una gráfica en la que para todo $v \in V(G)$ se tiene que $ 2 \leq d_G(v)$ entonces $G$ contiene un ciclo.

\item Demostrar que si $G$ es una gráfica en la que $|V(G)| \leq |A(G)|$ entonces $G$ contiene un ciclo.

%\item Demostrar que si $G$ es un árbol con exactamente dos hojas entonces $G$ es una trayectoria.

\item Demostrar que si $G$ es un árbol tal que $k \leq \Delta(G)$ entonces $G$ tiene al menos $k$ hojas.

\item Demostrar que las siguientes proposiciones son equivalentes:
\begin{itemize}
\item $G$ es una gráfica conexa con $|V(G)|-1 = |A(G)|$.
\item $G$ es un bosque con $|V(G)|-1 = |A(G)|$.
\item $G$ es un árbol.
\end{itemize}

\item Demostrar que $G$ es un bosque si y solamente si $|A(G)| = |V(G)|-cc(G)$.

%\item Demostrar que si $G$ es un árbol entonces tiene un centro o dos centros que son adyacentes\footnote{Sea $G$ una gráfica. Definimos, para cualquier $v \in V(G)$, $\mathbb{A}_v = \{ k \in \mathbb N \; | \; k = d_G(v,x) \text{ con } x \in V(G)\}$ y sea $a_v = \max(\mathbb A_v)$.
%
%$v \in V(G)$ es un \textbf{centro de $G$} si y solamente si $v$ es un vértice que satisface con ser el vértice de $G$ que representa al $\min(\{ a_v \in \mathbb N \; | \; v \in V(G) \})$}.

\item Demostrar que una gráfica es conexa si y solamente si tiene un árbol generador.

\item Demostrar que todo árbol es una gráfica bipartita.
\end{enumerate}


\end{document}
