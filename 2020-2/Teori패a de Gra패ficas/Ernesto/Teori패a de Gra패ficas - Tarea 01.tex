\documentclass[10pt]{report}

\usepackage{amssymb}
\usepackage{amsmath}
\usepackage{amscd}
\usepackage{amsthm}
\usepackage[utf8]{inputenc}
\usepackage[spanish,mexico]{babel}
\usepackage{enumerate}
\usepackage[usenames]{color}
\numberwithin{section}{chapter}

\usepackage{pgf,tikz}
\usetikzlibrary{arrows}

\usepackage{multicol}

\usepackage{graphicx}
\usepackage{subfigure}

\usetikzlibrary{knots,hobby,decorations.pathreplacing,shapes.geometric,calc}
\tikzset{knot diagram/every strand/.append style={ultra thick,red}, show curve controls/.style={postaction=decorate, decoration={show path construction, curveto code={\draw [blue, dashed](\tikzinputsegmentfirst) -- (\tikzinputsegmentsupporta) node [at end, draw, solid, red, inner sep=2pt]{}; \draw [blue, dashed] (\tikzinputsegmentsupportb) -- (\tikzinputsegmentlast) node [at start, draw, solid, red, inner sep=2pt]{} node [at end, fill, blue, ellipse, inner sep=2pt]{};}}}, show curve endpoints/.style={ postaction=decorate, decoration={show path construction, curveto code={\node [fill, blue, ellipse, inner sep=2pt] at (\tikzinputsegmentlast) {};}}}}

\usepackage[colorlinks=true, linkcolor=blue, urlcolor=red,
citecolor=green]{hyperref}

\voffset=-2cm
\hoffset=-3cm
\textwidth = 18cm
\textheight= 23 cm

\usepackage{iwona}
\usepackage{fancyhdr}
\pagestyle{fancy}
\fancyhf{}
\fancyhead[RE,LO]{\bfseries{Teoría de Gráficas}}
\fancyhead[LE,RO]{\bfseries{2020-2}}
\fancyfoot[RE,RO]{\bfseries{Febrero 2020}}
\fancyfoot[LE,LO]{\bfseries{Evaluación Parcial 01}}

\newcommand{\N}{\mathbb N}
\newcommand{\Q}{\mathbb Q}
\newcommand{\E}{\mathbb E}
\newcommand{\s}{\mathbb S}
\newcommand{\C}{\mathbb C}
\newcommand{\F}{\mathbb F}
\newcommand{\T}{\mathbb T}
\newcommand{\p}{\mathbb P}
\newcommand{\I}{\mathbb I}
\newcommand{\A}{\mathbb A}


\begin{document}
\begin{center}
\textcolor{blue}{\textbf{\large Guía de ejercicios para al Evaluación Parcial 01}}\\
\vspace{0.5 cm}
\textcolor{red}{\textbf{\large FECHA DE EXAMEN PARCIAL 01 \\ VIERNES 28-FEBRERO-2020\\ De 09:00 a 10:00 HORAS - Salón P-118}}
\end{center}

\textbf{Instrucciones}: La siguiente lista que fungirá como guía para el examen parcial, se recomienda resolver todos los ejercicios de la misma.

\vspace{1cm}


\begin{center}
\textcolor{blue}{\textbf{\large LISTA DE EJERCICIOS}}
\end{center}


\begin{enumerate}
\item Sea $G(V(G), A(G))$ y $H=(V(H), A(H))$ gráficas.

Demostrar que la relación $G \sim H$ si y solamente si $G \cong H$, es una relación de equivalencia.

\item ¿Cuántas gráficas distintas de orden cuatro existen salvo isomorfismo?

\item Demostrar que, salvo isomorfismo, para cualquier $n \in \N \setminus\{0\}$ la gráfica completa de orden $n$ es única.\footnote{Por ello, a la gráfica completa de orden $n$ la denotaremos como $K_n$.}

\item Demostrar que, salvo isomorfismo, para cualquier $\{n,m\} \subseteq \N \setminus\{0\}$ si $|X| = n$ y $|Y|=m$ entonces la gráfica bipartita completa $G[X,Y]$ es única.\footnote{Por ello, a la gráfica bipartita completa $G[X,Y]$ la denotaremos como $K_{|X|,|Y|}$.}

\item Demuestra que todo camino con extremos $x$ y $y$ contiene una trayectoria con extremos $x$ y $y$.

\item ¿Qué relación existe entre $K_1$ y $K_{1,1}$?

\item Demostrar que en toda gráfica se cumple que $|A(G)| \leq {|V(G)|\choose 2}$. ¿En qué caso se da la igualdad?

\item Considerar a $G[X,Y]$ para demostrar que:
\begin{itemize}
\item $|A(G[X,Y])| \leq |X||Y|$.
\item $4 |A(G[X,Y])| \leq |V(G[X,Y])|^2$.
\item ¿En qué caso se cumple que $4 |A(G[X,Y])| = |V(G[X,Y])|^2$?
\end{itemize}

\item Sea $G$ una gráfica. Considerar lo siguiente:
\begin{itemize}
\item $\delta (G) : = \min(\{d_G(x) \in \N \cup \{0\} \; | \; x \in V(G) \})$\footnote{A $\delta(G)$ se le conoce como el \textbf{gradro mínimo de $G$}.}
\item $\Delta (G) : = \max(\{d_G(x) \in \N \cup \{0\} \; | \; x \in V(G) \})$\footnote{A $\Delta(G)$ se le conoce como el \textbf{gradro máximo de $G$}.}
\end{itemize}
Demostrar que $$ \delta(G) \leq \frac{2 |A(G)|}{|V(G)|} \leq \Delta(G)$$


\item Demostrar que para una gráfica $G$ las siguientes son equivalentes:
\begin{itemize}
\item $G$ es conexa.
\item Para cualquier $\{x,y\} \subset V(G)$ con $x \neq y$ existe una trayectoria con extremos $x$ y $y$.
\item Existe un camino cerrado en $G$ que contiene a todos los vértices y a todas las aristas de $G$.
\end{itemize}

\item 

\end{enumerate}

\end{document}
