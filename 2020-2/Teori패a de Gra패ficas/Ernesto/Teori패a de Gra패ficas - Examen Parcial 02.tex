\documentclass[12pt]{report}

\usepackage{amssymb,amsmath}
\usepackage[utf8]{inputenc}
\usepackage[spanish,mexico]{babel}

\usepackage{iwona}

\voffset=-3cm
\hoffset=-3cm
\textwidth=19 cm
\textheight=26 cm

\newcommand{\R}{\mathbb R}
\newcommand{\Q}{\mathbb Q}
\newcommand{\E}{\mathbb E}
\newcommand{\s}{\mathbb S}
\newcommand{\C}{\mathbb C}
\newcommand{\F}{\mathbb F}
\newcommand{\T}{\mathbb T}
\newcommand{\p}{\mathbb P}
\newcommand{\I}{\mathbb I}
\newcommand{\A}{\mathbb A}

\begin{document}

\begin{center}
\textbf{\LARGE {TEORÍA DE GRÁFICAS}}
\end{center}

\begin{center}
\textbf{{\large 2020-2 (02 abril 2020)}}
\end{center}

\begin{center}
\textbf{{\large EXAMEN PARCIAL 02}}
\end{center}

{\bf INSTRUCCIONES:}
\begin{itemize}
\item Justificar y argumentar todos los resultados que se realicen.

\item De no haber recibido un correo electrónico por parte de los profesores argumentando que se ha obtenido el \textbf{Derecho a Examen}, no se considerará a revisión cualquier archivo que se envíe como \textbf{Examen Parcial 02}.

\item Resolver y enviar por correo electrónico, a ambos profesores, \textbf{cinco} de los siguientes ejercicios. De entregar mas de cinco ejercicios, se evaluarán todos los ejercicios pero no se considerará para el promedio el ejercicio de mayor puntaje.

\item La fecha límite de envío del \textbf{Examen Parcial 02} es el
\begin{center}
\textbf{Viernes 03 de abril de 2020 a las 12:00 horas}
\end{center}
No se considerará a revisión cualquier archivo que se envíe como \textbf{Examen Parcial 02} después de esta fecha y horario.

\end{itemize}

\begin{center}
\rule[0mm]{20cm}{0.2mm}
\end{center}

\begin{enumerate}

%\item
%\begin{enumerate}
%\item Demostrar que toda gráfica autocomplementaria\footnote{$G$ es una gráfica \textbf{autocomplementaria} si y solamente si $G \cong \overline G$.} es conexa.
%
%\item Demostrar que si $G$ es una gráfica autocomplementaria entonces $|V(G)| \equiv x \mod 4$ donde $x \in \{0,1\}$.
%
%\item Construir, para cada $k \in \mathbb N$, una gráfica autocomplementaria de orden $4k$.
%
%\item Construir, para cada $k \in \mathbb N$, una gráfica autocomplementaria de orden $4k+1$.

\item Demostrar que toda gráfica autocomplementaria con $4k+1$ vértices tiene un vértice de grado $2k$.
%\end{enumerate}

%\item Demostrar que la siguiente gráfica $G$ no es una gráfica de intervalos:
%\begin{itemize}
%\item $V(G)=\{P,Q,R,S,T,U\}$
%\item $A(G) = \{\{P,Q\},\{Q,R\},\{R,P\},\{Q,S\},\{S,T\},\{T,Q\},\{R,T\},\{T,U\},\{U,R\}\}$
%\end{itemize}

%\item Demostrar que si $G$ es una gráfica conexa de orden mayor que uno entonces
%\begin{itemize}
%\item $1 \leq c(G-\{v\}) \leq d_G(v)$ para cualquier $v \in V(G)$.
%\item $1 \leq c(G\setminus \{a\}) \leq 2$ para cualquier $a \in A(G)$.
%\end{itemize}

%\item Demostrar que toda gráfica disconexa puede expresarse como una suma de gráficas conexas.

%EXAMEN
\item Demostrar que si $G$ es una gráfica en la que para todo $v \in V(G)$ se tiene que $ 2 \leq d_G(v)$ entonces $G$ contiene un ciclo.

%EXAMEN
\item Demostrar que si $G$ es una gráfica donde $|V(G)| \leq |A(G)|$ entonces $G$ contiene un ciclo.

%\item Demostrar que si $G$ es una gráfica donde $2 \leq \delta(G)$ entonces $G$ contiene un ciclo de longitud al menos $\delta(G)+1$.

%\item Demostrar que las siguientes afirmaciones son equivalentes para una gráfica conexa $G$ de orden al menos tres:
%\begin{itemize}
%\item $G$ es un bloque\footnote{$G$ es un \textbf{bloque} si y solamente si $G$ es una gráfica conexa sin vértices de corte.}.
%\item Para cualquier $\{u,v\} \subseteq V(G)$ con $u \neq v$ existe un ciclo $C$ en $G$ tal que $\{u,v\} \subset V(C)$.
%\item Para cualquier $u \in V(G)$  y $a \in A(G)$  existe un ciclo $C$ en $G$ tal que $u \in V(C)$ y $a \in A(C)$.
%\item Para cualquier $\{a,b\} \subseteq A(G)$ con $a \neq b$ existe un ciclo $C$ en $G$ tal que $\{a,b\} \subset A(C)$.
%\end{itemize}

%%%SI
\item Demostrar que si $G$ es un bloque tal que $3 \leq |V(G)|$, $\{u,v\}\subseteq V(G)$ tal que $v \neq u$ y $T_{uv}$ es una trayectoria en $G$ con extremos $u$ y $v$ entonces existe una trayectoria $T_{uv}^{*}$ con extremos $u$ y $v$ tal que $V(T_{uv}) \cap V(T_{uv}^*) = \{u,v\}$.

%\item Sea $G$ una gráfica con cuatro bloques tal que $V(G)=\{v_1,v_2,v_3,v_4,v_5,v_6,v_7, v_8\}$. Demostrar que si para cualquier $1\leq i \leq 6$ se tiene que $v_i$ pertenece a exactamente un bloque de $G$ y que $v_7$ y $v_8$ pertenecen exactamente a dos bloques de $G$ entonces $G$ es disconexa.

%%% SÍ
\item Demostrar que si $v \in V(G)$ es vértice de corte en $G$ entonces $v$ no es vértice de corte en $\overline{G}$

%\item Sea $G$ una gráfica conexa con al menos un vértice de corte. Demostrar que $G$ contiene, por lo menos, dos bloques cada uno de los cuales contiene exactamente un vértice de corte de $G$\footnote{Los bloques de una gráfica que contienen exactamente un vértice de corte de dicha gráfica se llaman \textbf{bloques terminales de $G$}}.

%\item Sea $G$ una gráfica conexa con al menos un vértice de corte. Demostrar que $G$ contiene un vértice de corte $v$ con la propiedad de que, con a lo más una excepción, todos los bloques que contienen a $v$ son bloques terminales.

%\item Demostrar que si $G$ es una gráfica que no contiene ciclos pares entonces todo bloque de $G$ es isomorfo a $K_2$ o es un ciclo impar.

%\item Demostrar que el número de bloques de una gráfica $G$ es igual a
%$$c(G)+ \sum_{v\in V(G)}(b(v)-1)$$
%donde $b(v)$ denota al número de bloques de $G$ que contienen a $v$.

%%%SÍ
\item Demostrar que si $G$ es una gráfica conexa con exactamente dos vértices que no son vértices de corte entonces $G$ es una trayectoria.

%\item Demostrar que en cualquier gráfica $G$ si $a\in A(G)$ entonces $c(G) \leq c(G \setminus \{a\}) \leq c(G)+1$.


\end{enumerate}


\vspace{2cm}


\end{document}
