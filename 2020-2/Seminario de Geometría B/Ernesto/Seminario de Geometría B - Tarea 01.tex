\documentclass[12pt]{report}

\usepackage{amssymb}
\usepackage{amsmath}
\usepackage{amscd}
\usepackage{amsthm}
\usepackage[utf8]{inputenc}
\usepackage[spanish,mexico]{babel}
\usepackage{enumerate}
\usepackage[usenames]{color}
\numberwithin{section}{chapter}

\usepackage{pgf,tikz}
\usetikzlibrary{arrows}

\usepackage{multicol}

\usepackage{graphicx}
\usepackage{subfigure}

\usetikzlibrary{knots,hobby,decorations.pathreplacing,shapes.geometric,calc}
\tikzset{knot diagram/every strand/.append style={ultra thick,red}, show curve controls/.style={postaction=decorate, decoration={show path construction, curveto code={\draw [blue, dashed](\tikzinputsegmentfirst) -- (\tikzinputsegmentsupporta) node [at end, draw, solid, red, inner sep=2pt]{}; \draw [blue, dashed] (\tikzinputsegmentsupportb) -- (\tikzinputsegmentlast) node [at start, draw, solid, red, inner sep=2pt]{} node [at end, fill, blue, ellipse, inner sep=2pt]{};}}}, show curve endpoints/.style={ postaction=decorate, decoration={show path construction, curveto code={\node [fill, blue, ellipse, inner sep=2pt] at (\tikzinputsegmentlast) {};}}}}

\usepackage[colorlinks=true, linkcolor=blue, urlcolor=red,
citecolor=green]{hyperref}

\voffset=-2cm
\hoffset=-2cm
\textwidth = 18cm
\textheight= 23 cm

\usepackage{iwona}
\usepackage{fancyhdr}
\pagestyle{fancy}
\fancyhf{}
\fancyhead[RE,LO]{\bfseries{Seminario de Geometría B}}
\fancyhead[LE,RO]{\bfseries{2020-2}}
\fancyfoot[RE,RO]{\bfseries{Marzo 2020}}
\fancyfoot[LE,LO]{\bfseries{Evaluación Parcial 01}}

\newcommand{\N}{\mathbb N}
\newcommand{\Q}{\mathbb Q}
\newcommand{\E}{\mathbb E}
\newcommand{\s}{\mathbb S}
\newcommand{\C}{\mathbb C}
\newcommand{\F}{\mathbb F}
\newcommand{\T}{\mathbb T}
\newcommand{\p}{\mathbb P}
\newcommand{\I}{\mathbb I}
\newcommand{\A}{\mathbb A}


\begin{document}
\begin{center}
\textcolor{blue}{\textbf{\large Guía de ejercicios para al Evaluación Parcial 01}}\\
\vspace{0.5 cm}
\textcolor{red}{\textbf{\large FECHA DE EVALUACIÓN PARCIAL 01 \\ VIERNES 03-MARZO-2020\\ 12:00 HORAS}}
\end{center}

\textbf{Instrucciones}: Resolver y enviar por correo electrónico a ambos profesores la siguiente lista de ejercicios. Argumentar detalladamente todos los resultados.

\vspace{1cm}


\begin{center}
\textcolor{blue}{\textbf{\large LISTA DE EJERCICIOS}}
\end{center}


\begin{enumerate}
\item Definir el concepto de orientación de manera que la reflexión por cualquier recta en el plano sea una isometría que invierte la orientación.

\item Determinar el conjunto de puntos fijos de una reflexión, una rotación, una traslación y un deslizamiento.

\item Con los resultados de los ejercicios anteriores, clasificar al conjunto de isometrías de acuerdo al conjunto de puntos fijos y si preservan o invierten la orientación. Expresar ducha clasificación en una tabla.

\item Sean $\{x, y\}\subseteq \mathbb{R}^2$ con $x \neq y$. Demostrar que existe una traslación, una reflexión, una rotación y un deslizamiento de manera que la imagen de $x$ bajo éstas isometrías es $y$\footnote{Esto muestra que el grupo de las isometrías actúa transitivamente en el plano euclideano}. ¿Es única la traslación, la reflexión, la rotación y el deslizamiento con dicha propiedad?. En caso de que no, indicar cuántas hay.

\item Demostrar que $\forall f\in Iso(\E^2)$ existe $g \in Stab_{Iso(\E^2)}((0,0))$ y $t_u \in T(E^2)$ tal que $f= t_u \circ g$.

\item Dar una biyección entre $Stab_{Iso(\E^2)}((0,0))$ y el conjunto $S^2 \times Z_2$.

\item Demostrar que existe una función biyectiva $\varphi: Iso(\E^2) \to S^2 \times Z_2 \times R^2$.

\item Encontrar una operación $*$ en $S^2 \times Z_2 \times R^2$ tal que 
$$(Iso(\E^2), \circ, Id_{Iso(\E^2)}) \cong( S^2 \times Z_2 \times R^2, *, e)$$
donde $e$ es el elemento neutro de $S^2 \times Z_2 \times R^2$ con respecto a $*$


\end{enumerate}

\end{document}
