\documentclass[12pts]{report}


\usepackage{amssymb}
\usepackage{amsmath}
\usepackage{amscd}
\usepackage{amsthm}
\usepackage[utf8]{inputenc}
\usepackage[spanish,mexico]{babel}
\usepackage{enumerate}
\usepackage[usenames]{color}


\usepackage{pgf,tikz}
\usetikzlibrary{arrows}

\usepackage[colorlinks=true, linkcolor=blue, urlcolor=red,
citecolor=green]{hyperref}

\voffset=-2cm
\hoffset=-2cm
\textwidth = 18cm
\textheight= 23 cm

\usepackage{iwona}
\usepackage{fancyhdr}
\pagestyle{fancy}
\fancyhf{}
\fancyhead[RE,LO]{\bfseries{Seminario de Geometría}}
\fancyhead[LE,RO]{\bfseries{2020-2}}
\fancyfoot[RE,RO]{\bfseries{Marzo 2020}}
\fancyfoot[LE,LO]{\bfseries{Tarea 01}}

\newcommand{\R}{\mathbb R}
\newcommand{\Q}{\mathbb Q}
\newcommand{\E}{\mathbb E}
\newcommand{\s}{\mathbb S}
\newcommand{\C}{\mathbb C}
\newcommand{\F}{\mathbb F}
\newcommand{\T}{\mathbb T}
\newcommand{\p}{\mathbb P}
\newcommand{\I}{\mathbb I}
\newcommand{\A}{\mathbb A}
\newcommand{\h}{\mathbb H}

\begin{document}
\begin{center}
\textbf{\LARGE {SEMINARIO DE GEOMETRÍA B}}
\end{center}

\begin{center}
\textbf{\large MARZO 2020}\\
\end{center}

\begin{center}
\textbf{{\large TAREA}}
\end{center}

{\bf INSTRUCCIONES: }
\begin{itemize}
\item Justificar y argumentar todos los resultados que se realicen
\end{itemize}

\begin{center}
\rule[0mm]{20cm}{0.2mm}
\end{center}

%\vspace{1cm}
\begin{enumerate}
\item Definir la orientación de una isometría y  usando esa definición demostrar que las reflexiones invierten orientación.

\item Determinar el conjunto de puntos fijos de una reflexión, una rotación, una traslación y un deslizamiento.

\item Sean $\{x, y\}\subset \R^2$. Demostrar que existe una traslación, una reflexión, una rotación y un deslizamiento que mande $x$ en $y$.¿Es única la traslación, la reflexión, la rotación y el deslizamiento con dicha propiedad?. En caso de que no sea única indicar cuantas hay.

\item Demostrar que $\forall f\in Iso(\E^2)$ entonces existe $g\in Stab_{Iso(\E^2)}(0)$ y $t_u \in T(E^2)$ tal que
$$f= t_u \circ g$$

\item Justificar porque el espacio del $ Stab_{Iso(\E^2)}(0)$ es $S^2 \times Z_2$

\item Demostrar que existe una función biyectiva $\varphi: Iso(\E^2) \to S^2 \times Z_2 \times R^2$.

\item Encontrar una operación $*$ en $S^2 \times Z_2 \times R^2$ tal que 
$$(Iso(\E^2), \circ, Id_{Iso(\E^2)}) \cong( S^2 \times Z_2 \times R^2, *, e)$$
donde $e$ es el elemento neutro de $S^2 \times Z_2 \times R^2$ con respecto a $*$
\end{enumerate}



\end{document}
