\documentclass[12pt]{report}

\usepackage{amssymb,amsmath}
\usepackage[utf8]{inputenc}
\usepackage[spanish,mexico]{babel}

\usepackage{iwona}

\voffset=-3cm
\hoffset=-2.5cm
\textwidth=18 cm
\textheight=26 cm

\newcommand{\R}{\mathbb R}
\newcommand{\Z}{\mathbb Z}
\newcommand{\N}{\mathbb N}
\newcommand{\Q}{\mathbb Q}
\newcommand{\E}{\mathbb E}
\newcommand{\s}{\mathbb S}
\newcommand{\C}{\mathbb C}
\newcommand{\F}{\mathbb F}
\newcommand{\T}{\mathbb T}
\newcommand{\p}{\mathbb P}
\newcommand{\I}{\mathbb I}
\newcommand{\A}{\mathbb A}

\begin{document}

\begin{center}
\textbf{\LARGE {TEORÍA DE GRÁFICAS}}
\end{center}

\begin{center}
\textbf{{\large 2020-2 (29 mayo 2020)}}
\end{center}

\begin{center}
\textbf{{\large EXAMEN PARCIAL 04}}
\end{center}

{\bf INSTRUCCIONES:}
\begin{itemize}
\item Justificar y argumentar todos los resultados que se realicen.

\item De no haber recibido un correo electrónico por parte de los profesores argumentando que se ha obtenido el \textbf{Derecho a Examen}, no se considerará a revisión cualquier archivo que se envíe como \textbf{Examen Parcial 04}.

\item Resolver y enviar por correo electrónico, a ambos profesores, \textbf{cinco} de los siguientes ejercicios a libre elección. De entregar mas de cinco ejercicios, se evaluarán todos los ejercicios pero no se considerará para el promedio el ejercicio de mayor puntaje.

\item La fecha límite de envío del \textbf{Examen Parcial 04} es el
\begin{center}
\textbf{Viernes 29 de mayo de 2020 a las 12:00 horas}
\end{center}
No se considerará a revisión cualquier archivo que se envíe como \textbf{Examen Parcial 04} después de esta fecha y horario.

\end{itemize}

\begin{center}
\rule[0mm]{20cm}{0.2mm}
\end{center}

\begin{enumerate}
%\item Dada $k \in \N \setminus \{0\}$, encontrar una gráfica $G$ que sea $k$-conexa puntualmente en el que el conjunto de corte desconecte a $G$ en más de dos componentes conexas.
%
\item Demostrar que si $G$ es una gráfica tal que $|V(G)|-2 \leq \delta(G)$ entonces $\kappa(G) < \delta(G)$.
%
%\item Demostrar que si $G$ es una gráfica en la que $\frac{|V(G)|+k-2}{2} \leq \delta(G)$ entonces $G$ es $k$-conexa puntualmente.

\item Demostrar que si $G$ es una gráfica $k$-conexa puntualmente entonces $G+K_1$ es una gráfica $(k+1)$-conexa puntualmente.

%\item Demostrar que si $G$ es una gráfica $k$-conexa linealmente ($k \in \N \setminus \{0\}$) y $A \subseteq A(G)$ tal que $|A| =k$ entonces $c(G\setminus A) \leq 2$.

\item Demostrar que si $G$ es una gráfica $k$-conexa linealmente entonces $ k |V(G)| \leq 2 |A(G)|$.

%\item Demostrar que si $G$ es una gráfica en la que $|V(G)| \leq \delta(G)+2$ entonces $\lambda(G) = \delta(G)$.

\item Encontrar una gráfica $G$ tal que $\left[\frac{|V(G)|-2}{2}\right] = \delta(G)$ y $\lambda(G) < \delta(G)$.

\item Demostrar que si $G$ es una gráfica $3$-regular entonces $\kappa(G) = \lambda(G)$.

%\item Sean $\{a,b,c\} \subseteq \N \setminus \{0\}$, tales que $a \leq b \leq c$. Construir una gráfica $G$ tal que $\kappa(G) =a$, $\lambda (G) = b$ y $\delta(G) =c$.

\item Construir una gráfica $G$ tal que $\kappa(G) =3$, $\lambda (G) = 4$ y $\delta(G) =5$.

%\item Demostrar que si $G$ es una gráfica $k$-regular y $\kappa(G)=1$ entonces $\lambda(G) \leq \left[ \frac{k}{2}\right]$\footnote{Se define, para cualquier $x \in \R$, $[x] := \max(\{z \in \Z \; | \; z \leq x\})$.}.
\end{enumerate}


\end{document}
