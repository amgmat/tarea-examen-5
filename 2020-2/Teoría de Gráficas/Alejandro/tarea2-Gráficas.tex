\documentclass[12pts]{report}


\usepackage{amssymb}
\usepackage{amsmath}
\usepackage{amscd}
\usepackage{amsthm}
\usepackage[utf8]{inputenc}
\usepackage[spanish,mexico]{babel}
\usepackage{enumerate}
\usepackage[usenames]{color}


\usepackage{pgf,tikz}
\usetikzlibrary{arrows}

\usepackage[colorlinks=true, linkcolor=blue, urlcolor=red,
citecolor=green]{hyperref}

\voffset=-2cm
\hoffset=-2cm
\textwidth = 18cm
\textheight= 23 cm

\usepackage{iwona}
\usepackage{fancyhdr}
\pagestyle{fancy}
\fancyhf{}
\fancyhead[RE,LO]{\bfseries{Teoría de Gráficas}}
\fancyhead[LE,RO]{\bfseries{2020-2}}
\fancyfoot[RE,RO]{\bfseries{Marzo 2020}}
\fancyfoot[LE,LO]{\bfseries{Tarea 02}}

\newcommand{\R}{\mathbb R}
\newcommand{\Q}{\mathbb Q}
\newcommand{\E}{\mathbb E}
\newcommand{\s}{\mathbb S}
\newcommand{\C}{\mathbb C}
\newcommand{\F}{\mathbb F}
\newcommand{\T}{\mathbb T}
\newcommand{\p}{\mathbb P}
\newcommand{\I}{\mathbb I}
\newcommand{\A}{\mathbb A}
\newcommand{\h}{\mathbb H}

\begin{document}
\begin{center}
\textbf{\LARGE {TEORÍA DE GRÁFICAS}}
\end{center}

\begin{center}
\textbf{\large MARZO 2020}\\
\end{center}

\begin{center}
\textbf{{\large TAREA}}
\end{center}

{\bf INSTRUCCIONES: }
\begin{itemize}
\item Justificar y argumentar todos los resultados que se realicen
\end{itemize}

\begin{center}
\rule[0mm]{20cm}{0.2mm}
\end{center}

%\vspace{1cm}
\begin{enumerate}
\item Sea $G$ una gráfica. Si para todo $v\in V(G)$ se tiene que $d_G(v)\geq 2$ entonces $G$ contiene un ciclo.


\item Sea $G$ una gráfica. Si $G$ es un bloque tal que $|V(G)|\geq 3$ y $\{u,v\}\subset V(G)$ tal que $v\neq u$. Dada $T_{uv}$ una trayectoria con extremos $u$ y $v$ en $G$. ¿Siempre existe una trayectoria $T_{uv}^{*}$ con extremos $u$ y $v$ tal que sea ajena a $T_{uv}$ excepto en los vértices $u$ y $v$?

\item Sea $G$ una gráfica con cuatro bloques tal que $V(G)=\{v_1,v_2,\dots, v_8\}$. Supongamos que todo $v_i\; 1\leq i \leq 6$ está en exactamente en un bloque y que $v_7$ y $v_8$ pertenecen exactamente a dos bloques. Demostrar que $G$ es disconexa.

\item Sean $G$ una gráfica y $v\in V(G)$. Si $v$ es vértice de corte en $G$, entonces $v$ no es vertice de corte en $\overline{G}$

\item Sea $G$ una gráfica conexa con uno o más vértices de corte. Demostrar que $G$ contiene por lo menos dos bloques, cada uno de los cuales contiene exactamente un vértice de corte de $G$. (Los bloques de una gráfica que contienen exactamente un vértice de corte de dicha gráfica se llaman BLOQUES TERMINALES de G).


\item Sea $G$ una gráfica conexa con al menos un vértice de corte. Demuestra que $G$ contiene un vértice de corte $v$ con la propiedad de que, con a lo más una excepción, todos los bloques que contienen a $v$ son bloques terminales.
\item Sea $G$ es una gráfica. Si $G$ no tiene ciclos pares, entonces todo bloque de $G$ es $K_2$ o un ciclo impar.

\item Sea $G$ una gráfica. Demuestra que el número de bloques de una gráfica $G$ es igual a $c(G)+ \sum_{v\in V(G)}(b(v)-1)$ donde $b(v)$ es el número de bloques de $G$ que contienen a $v$.
\item Sea $G$ una gráfica. Demuestra que si $G$ es conexa tal que tiene exactament dos vértices que no son de corte entonces $G$ es una trayectoria.
\item Sea $G$ es una gráfica. Demuestra que si $a\in A(G)$, entonces $$c(G)\leq c(G\backslash \{a\}) \leq c(G)+1$$
\end{enumerate}



\end{document}
