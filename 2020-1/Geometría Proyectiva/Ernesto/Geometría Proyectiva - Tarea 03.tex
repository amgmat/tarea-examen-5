\documentclass[12pt]{report}

\usepackage{amssymb}
\usepackage{amsmath}
\usepackage{amscd}
\usepackage{amsthm}
\usepackage[utf8]{inputenc}
\usepackage[spanish,mexico]{babel}
\usepackage{enumerate}
\usepackage[usenames]{color}
\numberwithin{section}{chapter}

\usepackage{pgf,tikz}
\usetikzlibrary{arrows}

\usepackage{multicol}

\usepackage{graphicx}
\usepackage{subfigure}

\usetikzlibrary{knots,hobby,decorations.pathreplacing,shapes.geometric,calc}
\tikzset{knot diagram/every strand/.append style={ultra thick,red}, show curve controls/.style={postaction=decorate, decoration={show path construction, curveto code={\draw [blue, dashed](\tikzinputsegmentfirst) -- (\tikzinputsegmentsupporta) node [at end, draw, solid, red, inner sep=2pt]{}; \draw [blue, dashed] (\tikzinputsegmentsupportb) -- (\tikzinputsegmentlast) node [at start, draw, solid, red, inner sep=2pt]{} node [at end, fill, blue, ellipse, inner sep=2pt]{};}}}, show curve endpoints/.style={ postaction=decorate, decoration={show path construction, curveto code={\node [fill, blue, ellipse, inner sep=2pt] at (\tikzinputsegmentlast) {};}}}}

\usepackage[colorlinks=true, linkcolor=blue, urlcolor=red,
citecolor=green]{hyperref}

\voffset=-2cm
\hoffset=-2.25cm
\textwidth = 18cm
\textheight= 23 cm

\usepackage{iwona}
\usepackage{fancyhdr}
\pagestyle{fancy}
\fancyhf{}
\fancyhead[RE,LO]{\bfseries{Geometría Proyectiva}}
\fancyhead[LE,RO]{\bfseries{2020-1}}
\fancyfoot[RE,RO]{\bfseries{Octubre 2019}}
\fancyfoot[LE,LO]{\bfseries{Evaluación Parcial 03}}

\newcommand{\R}{\mathbb R}
\newcommand{\Q}{\mathbb Q}
\newcommand{\E}{\mathbb E}
\newcommand{\s}{\mathbb S}
\newcommand{\C}{\mathbb C}
\newcommand{\F}{\mathbb F}
\newcommand{\T}{\mathbb T}
\newcommand{\p}{\mathbb P}
\newcommand{\I}{\mathbb I}
\newcommand{\A}{\mathbb A}


\begin{document}
\begin{center}
\textcolor{blue}{\textbf{\large Guía de ejercicios para la Evaluación Parcial 03}}
\end{center}

\begin{center}
\textcolor{red}{\textbf{\large EVALUACIÓN PARCIAL 03\\
LUNES 14 AL LUNES 21 DE OCTUBRE DE 2019\\
De 19:00 a 20:00 HORAS - Salón P-213}}
\vspace{0.5 cm}
\end{center}

\textbf{Instrucciones}: La tercera evaluación consistirá en resolver todos los ejercicios de la siguiente lista y solamente se evaluarán cuatro de la siguiente manera:
\begin{itemize}
\item El \textbf{LUNES 21 DE OCTUBRE DE 2019 A LAS 19:00 HORAS} se entregarán por escrito tres de los ejercicios. Dichos ejercicios deberán elegirse a libre albedrío.
\item En la semana que comprende del \textbf{LUNES 14 AL VIERNES 18 DE OCTUBRE DE 2019} en el horario de clase se elegirá al azar el cuarto ejercicio que deberán exponer con lujo de detalle al grupo y que no podrá ser uno de los tres ejercicios que se entregarán.\\
\end{itemize}

\begin{enumerate}
\item 
\begin{enumerate}
\item Sea $l$ una recta en un plano y $A \in l$. Construir una proyectividad $\phi:l \to l$ tal que $\phi(A)=A$ y para cualquier $X \in l\setminus\{A\}$ se tenga que $\phi(X) \neq X$.
\item Sea $L$ un punto en el plano, $\{a,b\} \subseteq \Omega_L$. Construir una proyectividad $\phi:\Omega_L \to \Omega_L$ tal que para $x \in \{a,b\}$ se cumpla que $\phi(x)=x$ y para cualquier $p \in \Omega_L \setminus\{a,b\}$, $\phi(p) \neq p$.
\end{enumerate}

\item 
\begin{enumerate}
\item Sea $l$ una recta en un plano, $\{A,B\} \subseteq l$. Construir una proyectividad $\phi:l \to l$ tal que para $X \in \{A,B\}$ se cumpla que $\phi(X)=X$ y para cualquier $P \in l \setminus\{A,B\}$, $\phi(P) \neq P$.
\item Sea $L$ un punto en un plano y $a \in \Omega_L$. Construir una proyectividad $\phi:\Omega_L\to \Omega_L$ tal que $\phi(a)=a$ y para cualquier $x \in \Omega_L\setminus\{a\}$ se tenga que $\phi(x) \neq x$.
\end{enumerate}

\item Demostrar que si $\psi: \Omega_L \to \Omega_L$ es una perspectividad entonces $\psi= Id_{\Omega_L}$. 

\item Demostrar que si $\alpha: \Omega_L \to \Omega_M$ y $\beta: \Omega_M \to \Omega_L$ son perspectividades entonces $\beta \circ \alpha$ es una proyectividad con al menos dos rectas fijas.

\item Sea $\{A,B,C,D,E,F\} \subseteq l$. Demostrar que existe una proyectividad $\psi:l \to l$ tal que \linebreak $AECF\;\overset{\psi}{\overline{\wedge}}\; BDCF$ si y solamente si $(AD)(BE)(CF)$ es un conjunto cuadrangular.

\item Demostrar que toda proyectividad $\psi:l \to l$ es composición de a lo más tres perspectividades.

\item Demostrar que si $\psi:l \to l$ es una proyectividad elíptica entonces $\psi$ es composición de tres perspectividades.

\item Demostrar que si $\alpha: \Omega_L \to \Omega_L$ y $\beta:\Omega_L \to \Omega_L$ son  proyectividades parabólicas tales que $\alpha(a)=a=\beta(a)$ entonces $\alpha \circ \beta$ es una proyectividad parabólica o $\alpha \circ \beta = Id_{\Omega_L}$.

\item Sea $\phi:l \to l$ una proyectividad tal que $\psi(A)=A$ y para algún $B \in l\setminus\{A\}$, $\phi(B)=B'$, $\phi^2(B)=B''$. Demostrar que si $H(B',A;B,B'')$ entonces $\phi$ es una proyectividad parabólica.

\end{enumerate}
\end{document}