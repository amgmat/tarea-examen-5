\documentclass[12pt]{report}

\usepackage{amssymb}
\usepackage{amsmath}
\usepackage{amscd}
\usepackage{amsthm}
\usepackage[utf8]{inputenc}
\usepackage[spanish,mexico]{babel}
\usepackage{enumerate}
\usepackage[usenames]{color}
\numberwithin{section}{chapter}

\usepackage{pgf,tikz}
\usetikzlibrary{arrows}

\usepackage{multicol}

\usepackage{graphicx}
\usepackage{subfigure}

\usetikzlibrary{knots,hobby,decorations.pathreplacing,shapes.geometric,calc}
\tikzset{knot diagram/every strand/.append style={ultra thick,red}, show curve controls/.style={postaction=decorate, decoration={show path construction, curveto code={\draw [blue, dashed](\tikzinputsegmentfirst) -- (\tikzinputsegmentsupporta) node [at end, draw, solid, red, inner sep=2pt]{}; \draw [blue, dashed] (\tikzinputsegmentsupportb) -- (\tikzinputsegmentlast) node [at start, draw, solid, red, inner sep=2pt]{} node [at end, fill, blue, ellipse, inner sep=2pt]{};}}}, show curve endpoints/.style={ postaction=decorate, decoration={show path construction, curveto code={\node [fill, blue, ellipse, inner sep=2pt] at (\tikzinputsegmentlast) {};}}}}

\usepackage[colorlinks=true, linkcolor=blue, urlcolor=red,
citecolor=green]{hyperref}

\voffset=-2cm
\hoffset=-2.25cm
\textwidth = 18cm
\textheight= 23 cm

\usepackage{iwona}
\usepackage{fancyhdr}
\pagestyle{fancy}
\fancyhf{}
\fancyhead[RE,LO]{\bfseries{Geometría Proyectiva}}
\fancyhead[LE,RO]{\bfseries{2020-1}}
\fancyfoot[RE,RO]{\bfseries{Noviembre 2019}}
\fancyfoot[LE,LO]{\bfseries{Reposición Evaluación Parcial 03}}

\newcommand{\R}{\mathbb R}
\newcommand{\Q}{\mathbb Q}
\newcommand{\E}{\mathbb E}
\newcommand{\s}{\mathbb S}
\newcommand{\C}{\mathbb C}
\newcommand{\F}{\mathbb F}
\newcommand{\T}{\mathbb T}
\newcommand{\p}{\mathbb P}
\newcommand{\I}{\mathbb I}
\newcommand{\A}{\mathbb A}


\begin{document}
\begin{center}
\textcolor{blue}{\textbf{\large Reposición Evaluación Parcial 03}}\\
\vspace{0.5 cm}
\textcolor{red}{\textbf{\large FECHA DE ENTREGA \\ MIÉRCOLES 20-NOVIEMBRE-2019\\ De 17:00 a 19:00 HORAS - Salón P-213}}
\end{center}

\textbf{Instrucciones}: Resolver y entregar tres de los siguientes cuatro ejercicios. De entregar más de tres ejercicios se anulará el ejercicio de mayor puntaje.

\begin{enumerate}
\item Sea $l$ una recta en un plano, $\{A,B\} \subseteq l$. Construir una proyectividad $\phi:l \to l$ tal que para $X \in \{A,B\}$ se cumpla que $\phi(X)=X$ y para cualquier $P \in l \setminus\{A,B\}$, $\phi(P) \neq P$.

\item Sea $\{A,B,C,D,E,F\} \subseteq l$. Demostrar que existe una proyectividad $\psi:l \to l$ tal que \linebreak $AECF\;\overset{\psi}{\overline{\wedge}}\; BDCF$ si y solamente si $(AD)(BE)(CF)$ es un conjunto cuadrangular.

\item Demostrar que toda proyectividad $\psi:l \to l$ es composición de a lo más tres perspectividades.

\item Demostrar que si $\psi:l \to l$ es una proyectividad elíptica entonces $\psi$ es composición de tres perspectividades.

\end{enumerate}
\end{document}
