\documentclass[12pt]{report}

\usepackage{amssymb}
\usepackage{amsmath}
\usepackage{amscd}
\usepackage{amsthm}
\usepackage[utf8]{inputenc}
\usepackage[spanish,mexico]{babel}
\usepackage{enumerate}
\usepackage[usenames]{color}
\numberwithin{section}{chapter}

\usepackage{pgf,tikz}
\usetikzlibrary{arrows}

\usepackage{multicol}

\usepackage{graphicx}
\usepackage{subfigure}

\usetikzlibrary{knots,hobby,decorations.pathreplacing,shapes.geometric,calc}
\tikzset{knot diagram/every strand/.append style={ultra thick,red}, show curve controls/.style={postaction=decorate, decoration={show path construction, curveto code={\draw [blue, dashed](\tikzinputsegmentfirst) -- (\tikzinputsegmentsupporta) node [at end, draw, solid, red, inner sep=2pt]{}; \draw [blue, dashed] (\tikzinputsegmentsupportb) -- (\tikzinputsegmentlast) node [at start, draw, solid, red, inner sep=2pt]{} node [at end, fill, blue, ellipse, inner sep=2pt]{};}}}, show curve endpoints/.style={ postaction=decorate, decoration={show path construction, curveto code={\node [fill, blue, ellipse, inner sep=2pt] at (\tikzinputsegmentlast) {};}}}}

\usepackage[colorlinks=true, linkcolor=blue, urlcolor=red,
citecolor=green]{hyperref}

\voffset=-2cm
\hoffset=-2.25cm
\textwidth = 18cm
\textheight= 23 cm

\usepackage{iwona}
\usepackage{fancyhdr}
\pagestyle{fancy}
\fancyhf{}
\fancyhead[RE,LO]{\bfseries{Geometría Proyectiva}}
\fancyhead[LE,RO]{\bfseries{2020-1}}
\fancyfoot[RE,RO]{\bfseries{Noviembre 2019}}
\fancyfoot[LE,LO]{\bfseries{Reposición Evaluación Parcial 01}}

\newcommand{\R}{\mathbb R}
\newcommand{\Q}{\mathbb Q}
\newcommand{\E}{\mathbb E}
\newcommand{\s}{\mathbb S}
\newcommand{\C}{\mathbb C}
\newcommand{\F}{\mathbb F}
\newcommand{\T}{\mathbb T}
\newcommand{\p}{\mathbb P}
\newcommand{\I}{\mathbb I}
\newcommand{\A}{\mathbb A}


\begin{document}
\begin{center}
\textcolor{blue}{\textbf{\large Reposición Evaluación Parcial 01}}\\
\vspace{0.5 cm}
\textcolor{red}{\textbf{\large FECHA DE ENTREGA \\ MIÉRCOLES 20-NOVIEMBRE-2019\\ De 17:00 a 19:00 HORAS - Salón P-213}}
\end{center}

\textbf{Instrucciones}: Resolver y entregar tres de los siguientes cinco ejercicios. De entregar más de tres ejercicios se anularán los ejercicios de mayor puntaje.


\begin{enumerate}
\item Sean $l$ una recta y $\pi$ un plano en $\p^3$. Demostrar que si $l \not\subseteq \pi$ y $l \cap \pi \neq \emptyset$ entonces $|l\cap \pi|= 1$.

\item Demostrar que si tres triángulos están en perspectiva desde un mismo punto entonces los tres ejes de perspectiva, que determinan los triángulos por pares, son tres rectas concurrentes.

\item Demostrar que si dos cuadrángulos completos determinan el mismo conjunto cuadrangular entonces sus triángulos diagonales están en perspectiva.

\item Sea $\pi$ un plano en $\p^3$ y $\{A,B,C,D,E,F\} \subseteq \pi$ tales que $\{A,B,C\}$ y $\{D,E,F\}$ son puntos en posición general. Demostrar que $\pi_{ABC}=\pi_{DEF}$.

\item Construir un cuadrángulo completo que tenga a un triángulo dado como triángulo diagonal.
\end{enumerate}

\end{document}
