\documentclass[12pt]{report}

\usepackage{amssymb}
\usepackage{amsmath}
\usepackage{amscd}
\usepackage{amsthm}
\usepackage[utf8]{inputenc}
\usepackage[spanish,mexico]{babel}
\usepackage{enumerate}
\usepackage[usenames]{color}
\numberwithin{section}{chapter}

\usepackage{pgf,tikz}
\usetikzlibrary{arrows}

\usepackage{multicol}

\usepackage{graphicx}
\usepackage{subfigure}

\usetikzlibrary{knots,hobby,decorations.pathreplacing,shapes.geometric,calc}
\tikzset{knot diagram/every strand/.append style={ultra thick,red}, show curve controls/.style={postaction=decorate, decoration={show path construction, curveto code={\draw [blue, dashed](\tikzinputsegmentfirst) -- (\tikzinputsegmentsupporta) node [at end, draw, solid, red, inner sep=2pt]{}; \draw [blue, dashed] (\tikzinputsegmentsupportb) -- (\tikzinputsegmentlast) node [at start, draw, solid, red, inner sep=2pt]{} node [at end, fill, blue, ellipse, inner sep=2pt]{};}}}, show curve endpoints/.style={ postaction=decorate, decoration={show path construction, curveto code={\node [fill, blue, ellipse, inner sep=2pt] at (\tikzinputsegmentlast) {};}}}}

\usepackage[colorlinks=true, linkcolor=blue, urlcolor=red,
citecolor=green]{hyperref}

\voffset=-2cm
\hoffset=-2.25cm
\textwidth = 18cm
\textheight= 23 cm

\usepackage{iwona}
\usepackage{fancyhdr}
\pagestyle{fancy}
\fancyhf{}
\fancyhead[RE,LO]{\bfseries{Geometría Proyectiva}}
\fancyhead[LE,RO]{\bfseries{2020-1}}
\fancyfoot[RE,RO]{\bfseries{Septiembre 2019}}
\fancyfoot[LE,LO]{\bfseries{Evaluación Parcial 01}}

\newcommand{\R}{\mathbb R}
\newcommand{\Q}{\mathbb Q}
\newcommand{\E}{\mathbb E}
\newcommand{\s}{\mathbb S}
\newcommand{\C}{\mathbb C}
\newcommand{\F}{\mathbb F}
\newcommand{\T}{\mathbb T}
\newcommand{\p}{\mathbb P}
\newcommand{\I}{\mathbb I}
\newcommand{\A}{\mathbb A}


\begin{document}
\begin{center}
\textcolor{blue}{\textbf{\large Tarea-Examen de ejercicios para al Evaluación Parcial 01}}\\
\vspace{0.5 cm}
\textcolor{red}{\textbf{\large FECHA DE ENTREGA \\ VIERNES 13-SEPTIEMBRE-2019\\ De 17:00 a 19:00 HORAS - Salón O-223}}
\end{center}

\textbf{Instrucciones}: Resolver y entregar cuatro de los cinco ejercicios de solo una opción. De entregar más de cuatro ejercicios se anularán los ejercicios de mayor puntaje.

\vspace{1cm}


\begin{center}
\textcolor{blue}{\textbf{\large OPCIÓN A}}
\end{center}


\begin{enumerate}
\item Sean $\{P,Q\} \subseteq \p^3$ y $\pi$ un plano en $\p^3$. Demostrar que si $\{P,Q\} \subseteq \pi$ entonces $\overline{PQ}\subset \pi$.

%\item Sea $\pi$ un plano en $\p^3$ y $\{A,B,C,D,E,F\} \subseteq \pi$ tales que $\{A,B,C\}$ y $\{D,E,F\}$ son puntos en posición general. Demostrar que $\pi_{ABC}=\pi_{DEF}$.

\item Sean $l$ una recta y $\pi$ un plano en $\p^3$. Demostrar que si $l \not\subseteq \pi$ y $l \cap \pi \neq \emptyset$ entonces $|l\cap \pi|= 1$.

%\item Demostrar que si en cada recta en $\p^3$ inciden $n$ puntos distintos entonces en cada punto inciden $n$ rectas distintas.

\item Demostrar que existen cuatro puntos coplanaes que por ternas están en posición general.

%\item Demostrar que existen cuatro rectas coplanares que por ternas están en posición general.

\item Demostrar que si tres triángulos están en perspectiva desde un mismo punto entonces los tres ejes de perspectiva, que determinan los triángulos por pares, son tres rectas concurrentes.

%\item Demostrar que si tres triángulos están en perspectiva desde una misma recta entonces los tres centros de perspectiva, que determinan los triángulos por pares, son tres puntos colineales.

\item Demostrar que si dos cuadrángulos completos determinan el mismo conjunto cuadrangular entonces sus triángulos diagonales están en perspectiva.

%\item Construir un cuadrángulo completo que tenga a un triángulo dado como triángulo diagonal.
\end{enumerate}

\vspace{1cm}

\begin{center}
\textcolor{blue}{\textbf{\large OPCIÓN B}}
\end{center}


\begin{enumerate}
%\item Sean $\{P,Q\} \subseteq \p^3$ y $\pi$ un plano en $\p^3$. Demostrar que si $\{P,Q\} \subseteq \pi$ entonces $\overline{PQ}\subset \pi$.

\item Sea $\pi$ un plano en $\p^3$ y $\{A,B,C,D,E,F\} \subseteq \pi$ tales que $\{A,B,C\}$ y $\{D,E,F\}$ son puntos en posición general. Demostrar que $\pi_{ABC}=\pi_{DEF}$.

%\item Sean $l$ una recta y $\pi$ un plano en $\p^3$. Demostrar que si $l \not\subseteq \pi$ y $l \cap \pi \neq \emptyset$ entonces $|l\cap \pi|= 1$.

\item Demostrar que si en cada recta en $\p^3$ inciden $n$ puntos distintos entonces en cada punto inciden $n$ rectas distintas.

%\item Demostrar que existen cuatro puntos coplanaes que por ternas están en posición general.

\item Demostrar que existen cuatro rectas coplanares que por ternas están en posición general.

%\item Demostrar que si tres triángulos están en perspectiva desde un mismo punto entonces los tres ejes de perspectiva, que determinan los triángulos por pares, son tres rectas concurrentes.

\item Demostrar que si tres triángulos están en perspectiva desde una misma recta entonces los tres centros de perspectiva, que determinan los triángulos por pares, son tres puntos colineales.

%\item Demostrar que si dos cuadrángulos completos determinan el mismo conjunto cuadrangular entonces sus triángulos diagonales están en perspectiva.

\item Construir un cuadrángulo completo que tenga a un triángulo dado como triángulo diagonal.
\end{enumerate}

\end{document}
