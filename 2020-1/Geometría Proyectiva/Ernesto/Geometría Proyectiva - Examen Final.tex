\documentclass[12pt]{report}

\usepackage{amssymb,amsmath}
\usepackage[utf8]{inputenc}
\usepackage[spanish,mexico]{babel}

\usepackage{iwona}

\voffset=-3cm
\hoffset=-2cm
\textwidth=18 cm
\textheight=26 cm

\newcommand{\R}{\mathbb R}
\newcommand{\Q}{\mathbb Q}
\newcommand{\E}{\mathbb E}
\newcommand{\s}{\mathbb S}
\newcommand{\C}{\mathbb C}
\newcommand{\F}{\mathbb F}
\newcommand{\T}{\mathbb T}
\newcommand{\p}{\mathbb P}
\newcommand{\I}{\mathbb I}
\newcommand{\A}{\mathbb A}

\begin{document}

\begin{center}
\textbf{\LARGE {GEOMETRÍA PROYECTIVA}}
\end{center}

\begin{center}
\textbf{{\large 2020-1 (21 noviembre 2019)}}
\end{center}

\begin{center}
\textbf{{\large EXAMEN FINAL}}
\end{center}

{\bf INSTRUCCIONES}: Analizar cada uno de los siguientes ejercicios y al tener la idea clara de como resolverlos, indicar a los profesores para exponerlos. En la exposición se deberá justificar y argumentar todos los resultados que se realicen.

\begin{enumerate}

\item Construir un cuadrángulo completo que tenga a un triángulo dado como triángulo diagonal.

\item Sea $l$ una recta y $\{A,B,C,D\} \subseteq l$ un conjunto de cuatro puntos distintos. Demostrar que si $H(A,B;C,D)$ entonces $H(D,C; B,A)$.

\item Sea $\{A,B,C,D,E,F\} \subseteq l$. Demostrar que existe una proyectividad $\psi:l \to l$ tal que \linebreak $AECF\;\overset{\psi}{\overline{\wedge}}\; BDCF$ si y solamente si $(AD)(BE)(CF)$ es un conjunto cuadrangular.

\item Demostrar que toda involución con un punto fijo es involución hiperbólica.

\end{enumerate}

\end{document}
