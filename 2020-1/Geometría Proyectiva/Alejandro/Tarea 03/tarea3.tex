
  
\documentclass[12pts]{report}


\usepackage{amssymb}
\usepackage{amsmath}
\usepackage{amscd}
\usepackage{amsthm}
\usepackage[utf8]{inputenc}
\usepackage[spanish,mexico]{babel}
\usepackage{enumerate}
\usepackage[usenames]{color}


\usepackage{pgf,tikz}
\usetikzlibrary{arrows}

\usepackage[colorlinks=true, linkcolor=blue, urlcolor=red,
citecolor=green]{hyperref}

\voffset=-2cm
\hoffset=-2cm
\textwidth = 18cm
\textheight= 23 cm

\usepackage{iwona}
\usepackage{fancyhdr}
\pagestyle{fancy}
\fancyhf{}
\fancyhead[RE,LO]{\bfseries{Geometría Proyectiva}}
\fancyhead[LE,RO]{\bfseries{2020-1}}
\fancyfoot[RE,RO]{\bfseries{Octubre 2018}}
\fancyfoot[LE,LO]{\bfseries{Tarea 03}}

\newcommand{\R}{\mathbb R}
\newcommand{\Q}{\mathbb Q}
\newcommand{\E}{\mathbb E}
\newcommand{\s}{\mathbb S}
\newcommand{\C}{\mathbb C}
\newcommand{\F}{\mathbb F}
\newcommand{\T}{\mathbb T}
\newcommand{\p}{\mathbb P}
\newcommand{\I}{\mathbb I}
\newcommand{\A}{\mathbb A}
\newcommand{\h}{\mathbb H}

\begin{document}
\begin{center}
\textcolor{blue}{\textbf{\large Guia de ejercicios para la Evaluación Parcial 03 }}\\
\vspace{0.5 cm}
\end{center}

%\textbf{\large Instrucciones:}
%\begin{itemize}
%\item Se sugiere trabajar en equipo los ejercicios y expresar cualquier duda en clase o por correo electrónico.
%\end{itemize}

\begin{center}
%\textcolor{red}{\textbf{\large EXAMEN PARCIAL 02\\ VIERNES
%13-SEPTIEMBRE-2019\\ De 19:00 a 21:00 HORAS - Salón P-213}}
\end{center}

\begin{enumerate}
\item Sea $L$ un punto en $\p^3$. Demostrar que existe $\varphi : \Omega_L \to \Omega_L$ proyectividad que deja fijo dos rectas de $\Omega_L$.

\item Sea $L$ un punto en $\p^3$. Demostrar que existe $\psi : \Omega_L \to \Omega_L$ proyectividad que deja fijo una rectas de $\Omega_L$.

\item $L, M$ dos puntos en $\p^3$, $\varphi: \Omega_L\to \Omega_L$ proyectividad tal que  $\varphi =\beta\circ\alpha$, con $\alpha: \Omega_L\to\Omega_M$, $\beta: \Omega_M\to \Omega_L$ y $\Gamma = \{x\in\Omega_L\; | \; \varphi(x)= x\}$.
Demostrar que si $n=\overline{LM}$ es la única recta que cumple que $\alpha(n)=n=\beta(n)$  entonces $|\Gamma|\geq 2$

\item Sean $L, M$ en $\p^3$, $\varphi:\Omega_{L}\to\Omega_{L}$ y $\psi: \Omega_{L}\to \Omega_L$. Demostrar que si  $\overline{LM}=n$ es la única recta tal que \\
$\varphi(n)=n=\psi(n)$ entonces $\psi\circ\varphi$ es una proyectividad que deja fija únicamente a $n$ o $\psi\circ\varphi = Id_{\Omega_L}$


\end{enumerate}
\end{document}