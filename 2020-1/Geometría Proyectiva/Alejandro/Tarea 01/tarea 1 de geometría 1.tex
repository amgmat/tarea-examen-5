\documentclass[12pts]{report}


\usepackage{amssymb}
\usepackage{amsmath}
\usepackage{amscd}
\usepackage{amsthm}
\usepackage[utf8]{inputenc}
\usepackage[spanish,mexico]{babel}
\usepackage{enumerate}
\usepackage[usenames]{color}


\usepackage{pgf,tikz}
\usetikzlibrary{arrows}

\usepackage[colorlinks=true, linkcolor=blue, urlcolor=red,
citecolor=green]{hyperref}

\voffset=-2cm
\hoffset=-2cm
\textwidth = 18cm
\textheight= 23 cm

\usepackage{iwona}
\usepackage{fancyhdr}
\pagestyle{fancy}
\fancyhf{}
\fancyhead[RE,LO]{\bfseries{Geometría Proyectiva}}
\fancyhead[LE,RO]{\bfseries{2020-1}}
\fancyfoot[RE,RO]{\bfseries{Agosto 2018}}
\fancyfoot[LE,LO]{\bfseries{Tarea 01}}

\newcommand{\R}{\mathbb R}
\newcommand{\Q}{\mathbb Q}
\newcommand{\E}{\mathbb E}
\newcommand{\s}{\mathbb S}
\newcommand{\C}{\mathbb C}
\newcommand{\F}{\mathbb F}
\newcommand{\T}{\mathbb T}
\newcommand{\p}{\mathbb P}
\newcommand{\I}{\mathbb I}
\newcommand{\A}{\mathbb A}
\newcommand{\h}{\mathbb H}

\begin{document}
\begin{center}
\textcolor{blue}{\textbf{\large Guia de ejercicios para la Evaluación Parcial 01 }}\\
\vspace{0.5 cm}
\end{center}

%\textbf{\large Instrucciones:}
%\begin{itemize}
%\item Se sugiere trabajar en equipo los ejercicios y expresar cualquier duda en clase o por correo electrónico.
%\end{itemize}

\begin{center}
\textcolor{red}{\textbf{\large EXAMEN PARCIAL 01 \\ VIERNES
13-SEPTIEMBRE-2019\\ De 19:00 a 21:00 HORAS - Salón P-213}}
\end{center}

\begin{enumerate}
\item Sean $\{P,Q\} \subseteq \p^3$ y $\pi$ un plano en $\p^3$. Demostrar que si $\{P,Q\} \subseteq \pi$ entonces $\overline{PQ}\subset \pi$.

\item Sea $\pi$ un plano en $\p^3$ y $\{A,B,C,D,E,F\} \subseteq \pi$ tales que $\{A,B,C\}$ y $\{D,E,F\}$ son puntos en posición general. Demostrar que $\pi_{ABC}=\pi_{DEF}$

 \item Sean $l$ una recta y $\pi$ un plano en $\p^3$. Demostrar que si $l \not\subseteq \pi$ y $l \cap \pi \neq \emptyset$ entonces $|l\cap \pi|= 1$.

\item Demostrar que existen cuatro puntos coplanaes que están en posición general.

\item Demostrar que existen cuatro rectas coplanares que están en posición general.

\item Demostrar que si en cada recta de $\p^3$ tiene $n$ puntos distintos, entonces en cada punto inciden $n$ rectas distintas.

\item Demostrar que si tres triángulos están en perspectiva desde un mismo punto, entonces los ejes de perspectivan que determinan los triángulos por pares son concurrentes.

\item Demostrar que si tres triángulos están en perspectiva desde un mismo eje de perspectiva, entonces los centros de perspectiva que determinan los trángulos por pares son colineales.

\item Demostrar que si dos cuadrangulos completos determinan el mismo conjunto cuadrangular entonces sus triángulos diagonales están en perspectiva.
\end{enumerate}
\end{document}