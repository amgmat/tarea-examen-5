\documentclass[12pts]{report}


\usepackage{amssymb}
\usepackage{amsmath}
\usepackage{amscd}
\usepackage{amsthm}
\usepackage[utf8]{inputenc}
\usepackage[spanish,mexico]{babel}
\usepackage{enumerate}
\usepackage[usenames]{color}


\usepackage{pgf,tikz}
\usetikzlibrary{arrows}

\usepackage[colorlinks=true, linkcolor=blue, urlcolor=red,
citecolor=green]{hyperref}

\voffset=-2cm
\hoffset=-2cm
\textwidth = 18cm
\textheight= 23 cm

\usepackage{iwona}
\usepackage{fancyhdr}
\pagestyle{fancy}
\fancyhf{}
\fancyhead[RE,LO]{\bfseries{Geometría Proyectiva}}
\fancyhead[LE,RO]{\bfseries{2020-1}}
\fancyfoot[RE,RO]{\bfseries{Agosto 2018}}
\fancyfoot[LE,LO]{\bfseries{Tarea 01}}

\newcommand{\R}{\mathbb R}
\newcommand{\Q}{\mathbb Q}
\newcommand{\E}{\mathbb E}
\newcommand{\s}{\mathbb S}
\newcommand{\C}{\mathbb C}
\newcommand{\F}{\mathbb F}
\newcommand{\T}{\mathbb T}
\newcommand{\p}{\mathbb P}
\newcommand{\I}{\mathbb I}
\newcommand{\A}{\mathbb A}
\newcommand{\h}{\mathbb H}

\begin{document}
\begin{center}
\textcolor{blue}{\textbf{\large Guia de ejercicios para la Evaluación Parcial 01 }}\\
\vspace{0.5 cm}
\end{center}

\textbf{\large Instrucciones:}
\begin{itemize}
\item Se sugiere trabajar en equipo los ejercicios y expresar cualquier duda en clase o por correo electrónico.
\end{itemize}

\begin{center}
\textcolor{red}{\textbf{\large Axiomas de la Geometría Proyectiva}}
\end{center}

\begin{enumerate}
 \item Sean $P,Q$ dos puntos y $\pi$ un plano. Demostrar que si $\{P,Q\}\subset \pi$ entonces $\overline{PQ}\subset \pi$
 
 \item Sean $A,B,C,D,E,F$ seis puntos y $\pi_{ABC}$. Demostrar que si $\{D,E,F\}\subset \pi_{ABC}$ entonces $\pi_{ABC}=\pi_{DEF}$
 
\item Sean $l$ y $\pi$ un plano. Si $l\not\subset\pi$ y $l\cap\pi \neq \emptyset$ entonces $|l\cap \pi|= 1$.

\item Demostrar que existen cuatro puntos coplanaes que están en posición general.

\item Demostrar que existen cuatro rectas coplanares que están en posición general.
 
\end{enumerate}
\end{document}