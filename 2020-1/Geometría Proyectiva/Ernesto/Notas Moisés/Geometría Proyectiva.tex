\documentclass[12pt]{book}

\usepackage[T1]{fontenc}

\usepackage{fancyhdr}
\pagestyle{fancy}
\fancyhf{}
\renewcommand{\headrulewidth}{0.5pt}
\renewcommand{\footrulewidth}{0pt}
%
\renewcommand{\chaptermark}[1]{\markboth{#1}{}}
\renewcommand{\sectionmark}[1]{\markright{\thesection.\ #1}}
%
\fancyhead[LE,RO]{\bfseries\thepage}
\fancyhead[LO]{\bfseries\rightmark}
\fancyhead[RE]{\bfseries\leftmark}

\usepackage{amssymb}
\usepackage{amsmath}
\usepackage{amscd}
\usepackage{amsthm}
\usepackage[utf8]{inputenc}
\usepackage[spanish,mexico]{babel}
\usepackage{enumerate}
\usepackage{enumitem}
\usepackage{multicol}
\usepackage{pgf,tikz}
\usepackage{makeidx}
\usetikzlibrary{arrows}

\usepackage[colorlinks=true, linkcolor=blue, urlcolor=red,citecolor=green]{hyperref}

\usepackage[Glenn]{fncychap}
\ChNameVar{\bfseries\Large}
\ChNumVar{\Huge}
\ChTitleVar{\bfseries\Large}
\ChRuleWidth{0.5 pt}
\ChNameUpperCase
\ChTitleUpperCase

\theoremstyle{definition}
\newtheorem{teo}{\textcolor{red}{Teorema}}[chapter]
\newtheorem{df}[teo]{\textcolor{cyan}{Definición}}
\newtheorem{ej}[teo]{\textcolor{red}{Ejercicio}}
\newtheorem{prop}[teo]{\textcolor{red}{Proposición}}
\newtheorem{lema}[teo]{\textcolor{red}{Lema}}
\newtheorem{cor}[teo]{\textcolor{red}{Corolario}}
\newtheorem{obs}[teo]{\textcolor{purple}{Observación}}

\newenvironment{pba}{\noindent\textbf{\textcolor{blue}{Prueba:}}}{\begin{flushright}
$\square$ \end{flushright}}
\newenvironment{dem}{\noindent\textbf{\textcolor{blue}{Demostración}:}}{\begin{flushright}
\rule{1ex}{1ex} \end{flushright}}

\newcommand{\K}{\mathbb K}
\newcommand{\R}{\mathbb R}
\newcommand{\Z}{\mathbb Z}
\newcommand{\N}{\mathbb N}
\newcommand{\s}{\mathbb S}
\newcommand{\D}{\mathbb D}
\newcommand{\B}{\mathbb B}
\newcommand{\C}{\mathbb C}
\newcommand{\E}{\mathbb E}
\newcommand{\p}{\mathbb P}

\makeindex

\begin{document}
\title{\textbf{Geometría Proyectiva} \\ \vspace{1cm} \Large{Universidad Nacional Autónoma de México \\ Facultas de Ciencias}}
\date{2020-1}
\author{Ernesto Alejandro Vázquez Navarro}
\maketitle

\tableofcontents

\chapter*{Agradecimientos}
\addcontentsline{toc}{chapter}{Agradecimientos} 

Quiero agradecer infinitamente al \textbf{Ing. Moisés Rodríguez Salum} por compartirnos las notas que capturó del curso Geometría Proyectiva que impartí en la Facultad de Ciencias de la Universidad Nacional Autónoma de México en el semestre 2020-1 y que amablemente nos permitió modificarlas para su uso didáctico. \\

Hemos modificado el contenido de sus notas, por lo que nos hacemos responsables de cualquier error que se encuentre en estas notas y agradeceremos nos los hagan saber a ernestociencias@yahoo.com.mx. \\ \\

Ciudad de México, \today.

%%%AXIOMAS
\chapter{Axiomas de la geometría de $\p^3$}

\begin{enumerate}[label=\bf{Ax.\arabic*}]

\item \label{A1} Existe un punto y una recta que no son incidentes.

\item \label{A2} En cada recta inciden al menos tres puntos.

\item \label{A3} Para cualquier par de puntos distintos, $A \neq B$, existe una única recta en la que inciden que denotaremos como $\overline{AB}$.

\item \label{A4} Si $A,B,C$ y $D$ son puntos distintos tales que $\overline{AB} \cap \overline{CD} \neq \emptyset$ entonces $\overline{AC} \cap \overline{BD} \neq \emptyset$.

\item \label{A5} Para cualquier plano existe al menos un punto que no incide en el plano.

\item \label{A6} Cualesquiera dos planos distintos inciden en al menos dos puntos.

\item \label{A7} Los tres puntos diagonales de un cuadrángulo completo nunca son colineales.

\item \label{A8} Si una proyectividad deja fijos a tres puntos distintos de una recta entonces deja fijo a cualquier punto de la recta.

\end{enumerate}


%%%EL PRINCIPIO DE DUALIDAD
\chapter{El principio de dualidad}

%%%POLARIDAD
\section{Polaridad}\index{polaridad}

En esta sección veremos como el concepto de conjuntos armónicos (puntos o rectas) permiten construir una función, en un plano, en la que podamos asociar a algunos puntos (rectas) una recta (punto) en dicho plano.\\

%%% POLARIDAD TRILINEAL
\subsection{Polaridad trilineal}\index{polaridad!trilineal}

Consideremos un trilátero con lados $a$, $b$, $c$ y vértices
$$a \cap b =\{C\} \qquad ; \qquad b \cap c = \{A\} \qquad ; \qquad c \cap a = \{B\}$$

Así, para cualquier punto $P \notin a \cup b \cup c$ podemos considerar a las rectas
$$\overline{PA} = q \quad ; \quad \overline{PB} = r\quad ; \quad \overline{PC} = s$$

\begin{obs}
Notemos que $q$, $r$ y $s$ son distintas a las rectas $a$, $b$ y $c$, pues de no serlo, $P \in a \cup b \cup c$.
\end{obs}

\newpage

De esta manera, podemos considerar las siguientes intersecciones:
\begin{center}
\begin{multicols}{3}
$\overline{PA} \cap a = q \cap a = \{D\}$\\
$\overline{PA} \cap b = q \cap b = \{A\}$\\
$\overline{PA} \cap c = q \cap c = \{A\}$\\
$\overline{PB} \cap a = r \cap a = \{B\}$\\
$\overline{PB} \cap b = r \cap b = \{E\}$\\
$\overline{PB} \cap c = r \cap c = \{B\}$\\
$\overline{PC} \cap a = s \cap a = \{C\}$\\
$\overline{PC} \cap b = s \cap b = \{C\}$\\
$\overline{PC} \cap c = s \cap c = \{F\}$
\end{multicols}
\end{center}

Esto implica que
$$ \{B,C,D\}\subseteq a \qquad ; \qquad \{A,C,E\} \subseteq b \qquad ; \qquad \{A,B,F\} \subseteq c$$

De esta construcción notemos dos cosas:
\begin{enumerate}
\item\label{lineal1} Es posible construir puntos sobre los lados del trilátero que con los vértices del trilátero y los puntos en los que intersecan las rectas $\overline{PA}$, $\overline{PB}$ y $\overline{PC}$ sean conjuntos armónicos; es decir, construir a los puntos $G \in a$, $H \in b$ y $I \in c$ tales que
$$H(B,C;D,G) \quad ; \quad H(A,C;E,H) \quad ; \quad H(A,B;F,I)$$
Notemos que como $P\notin  a \cup b \cup c$ entonces $P, A, B, C$ son los vértices de un cuadrángulo completo cuyo triángulo diagonal es $\triangle DEF$. Puesto que $\{D,E,F\}$ es un conjunto de puntos en posición general (\ref{A7}), tenemos que las rectas $\overline{DE}=j$, $\overline{EF}=k$ y $\overline{FD}=l$ son distintas y considerar las siguientes intersecciones
$$j \cap c= \{I\} \quad ; \quad k \cap a = \{G\} \quad ; \quad l \cap b = \{H\}$$
es decir, 
$$\overline{DE} \cap \overline{AB} = \{I\} \quad ; \quad \overline{EF} \cap \overline{BC} = \{G\} \quad ; \quad \overline{FD} \cap \overline{CA} = \{H\}$$
De esta manera, tenemos que:
\begin{itemize}
\item $H(B,C;D,G)$:

Considerar al cuadrángulo completo $\square PAEF$ y $a$.

\item $H(A,C;E,H)$:

Considerar al cuadrángulo completo $\square PBFD$ y $b$.

\item $H(A,B;F,I)$:

Considerar al cuadrángulo completo $\square PCDE$ y $c$.
\end{itemize}

\item\label{lineal2} Notemos que:
\begin{itemize}
\item $D \in \overline{PA} = q \Leftrightarrow P \in \overline{AD} = q$
\item $E \in \overline{PB} = r  \Leftrightarrow P \in \overline{BE} = r$
\item $F \in \overline{PC} = s \Leftrightarrow P \in \overline{CF} = s$
\end{itemize}
en otras palabras,
$$\overline{AD} \cap \overline{BE} \cap \overline{CF} = \{P\}$$

Por lo tanto, se tiene que $\triangle ABC$ está en perspectiva desde $P$ con $\triangle DEF$. En virtud del Teorema de Desargues, tenemos que $\triangle ABC$ está en perspectiva desde una recta con $\triangle DEF$; es decir, la intersección de los lados correspondientes de los triángulos son tres puntos colineales.
\end{enumerate}

De esta manera, por~\ref{lineal1} y ~\ref{lineal2}, se tiene que $\{G,H, I\}$ es un conjunto de puntos colineales, donde
$$\overline{AB} \cap \overline{DE} = \{I\} \quad; \quad \overline{BC} \cap \overline{EF} = \{G\} \quad ; \quad \overline{CA} \cap \overline{FD} = \{H\}$$

Llamaremos $p$ a la recta en la que incide $\{G,H,I\}$.

De esta manera, $P$ es el \textbf{polo trilineal respecto a $\triangle abc$}\index{polaridad!trilineal!polo} de la recta $p$ y $p$ es la \textbf{polar trilineal respecto a $\triangle abc$}\index{polaridad!trilineal!polar} del punto $P$.


%%% POLARIDAD TRIPUNTUAL
\subsection{Polaridad tripuntual}\index{polaridad!tripuntual}

Consideremos un triángulo con vértices $A$, $B$, $C$ y lados
$$\overline{AB}= c \qquad ; \qquad \overline{BC} = a \qquad ; \qquad \overline{CA}= b $$

Así, para cualquier recta $p$ no incidente en $A$, $B$ ni $C$, podemos considerar los puntos
$$ p \cap a = \{Q\} \quad ; \quad p \cap b =\{R\} \quad ; \quad p \cap c = \{S\}$$

\begin{obs}
Notemos que $Q$, $R$ y $S$ son distintos a los puntos $A$, $B$ y $C$, pues de no serlo, $p$ incidiría en $A$, $B$ o $C$.
\end{obs}

De esta manera, podemos considerar las siguientes rectas:
\begin{center}
\begin{multicols}{3}
$\overline{QA}= d$\\
$\overline{QB}= a$\\
$\overline{QC}= a$\\
$\overline{RA}= b$\\
$\overline{RB}= e$\\
$\overline{RC}= b$\\
$\overline{SA}= c$\\
$\overline{SB}= c$\\
$\overline{SC}= f$
\end{multicols}
\end{center}

Esto implica que
$$ b \cap c \cap d = \{A\} \qquad ; \qquad a \cap c \cap e =\{B\} \qquad ; \qquad a \cap b \cap f = \{C\}$$

De esta construcción notemos dos cosas:
\begin{enumerate}
\item\label{puntual1} Es posible construir rectas incidentes en los vértices del triángulo que con los lados del trilángulo y las rectas en las que inciden $Q$, $R$ y $S$ sean conjuntos armónicos; es decir, construir a las rectas $g$, $h$, $i$ tales que $A \in g$, $B \in h$, $C\in i$ y  
$$H(b,c;d,g) \quad ; \quad H(a,c;e,h) \quad ; \quad H(a,b;f,i)$$
Notemos que como $p$ no incide en $A$, $B$ ni $C$ entonces $p, a, b, c$ son los lados de un cuadrilátero completo cuyo trilátero diagonal es $\triangle def$. Puesto que $\{d,e,f\}$ es un conjunto de rectas en posición general (Dual de \ref{A7}), tenemos que los puntos $d \cap e = \{J\}$, $e \cap f =\{K\}$ y $f \cap d =\{L\}$ son distintos y considerar las siguientes rectas
$$\overline{JC}= i\quad ; \quad \overline{KA} = g \quad ; \quad \overline{LB} =h$$
De esta manera, tenemos que:
\begin{itemize}
\item $H(b,c;d,g)$:

Considerar al cuadrilátero completo $\square paef$ y $A$.

\item $H(a,c;e,h)$:

Considerar al cuadrilátero completo $\square pbfd$ y $B$.

\item $H(a,b;f,i)$:

Considerar al cuadrilátero completo $\square pcde$ y $C$.
\end{itemize}


%%%%%
%%%%%
%%%%%
%%%%%
%%%%%

\item\label{puntual2} Notemos que:
\begin{itemize}
\item $d= \overline{QA}$ y $p \cap a = \{Q\} \Rightarrow d \cap a =\{Q\} \subseteq p$
\item $e= \overline{RB}$ y $p \cap b = \{R\} \Rightarrow e \cap b =\{R\} \subseteq p$
\item $f= \overline{SC}$ y $p \cap C = \{S\} \Rightarrow f \cap c =\{S\} \subseteq p$
\end{itemize}
en otras palabras,
$$a \cap d \subseteq p \quad ; \quad b \cap e \subseteq p \quad ; \quad c \cap f \subseteq p$$

Por lo tanto, se tiene que $\triangle abc$ está en perspectiva desde $p$ con $\triangle def$. En virtud del Teorema de Desargues, tenemos que $\triangle abc$ está en perspectiva desde un punto con $\triangle def$; es decir, las rectas en las que inciden vértices correspondientes son tres rectas concurrentes.
\end{enumerate}

De esta manera, por~\ref{puntual1} y ~\ref{puntual2}, se tiene que $g$ $h$ e $i$ es un conjunto de rectas concurrentes, donde
$$\overline{JC}= i \quad; \quad \overline{AK} = g \quad ; \quad \overline{BL} = h$$

Llamaremos $P$ al punto en donde inciden simultaneamente $g$, $h$ e $i$.

De esta manera, $p$ es la \textbf{polar tripuntual respecto a $\triangle ABC$}\index{polaridad!tripuntual!polar} del punto $P$ y $P$ es el \textbf{polo tripuntual respecto a $\triangle ABC$}\index{polaridad!tripuntual!polo} de la recta $p$.

\chapter{Dualidad en el plano}

%\begin{center}
Rectas $\Leftrightarrow$ Puntos\\
Concurrencia $\Leftrightarrow$ Colinealidad

%\begin{flushleft}
  Ejemplos de Dualidad\\
  Dado una recta inciden al menos tres puntos.\\
  Dual: Dado un punto inciden al menos tres rectas.
  
  %\begin{proof}[Prueba]
    Sea $P$ un punto en $\mathbb{P}^3$.\\
    Por axioma 1, existe una recta $l$ no incidente en $P$.

    Por axioma 2, existen tres puntos distintos \{$A$,$B$,$C$\} $\subseteq$ $l$.

    Por axioma 3, existen $\overline{AP}$, $\overline{BP}$, $\overline{CP}$.
  %  \begin{center}
    $\overline{AP} \neq \overline{BP}$\\
    $\overline{AP} \neq \overline{CP}$\\
    $\overline{BP} \neq \overline{CP}$\\

   % \begin{flushleft}
    Entonces existen tres rectas distintas que incidentes en $P$.
  %\end{proof}

%  \begin{flushleft}
  Dados dos puntos distintos, existe una única recta incidente en ellos.\\
  Dual: Dados dos rectas distintas coplanares, existe un único punto que es
  incidente en ambos.\\

%Dados $\{A$,$B$,$C$,$D\}$ puntos distintos tal que $\overline{AB} \cap \overline{CD} \neq \emptyset$ entonces $\overline{AD} \cap \overline{BC} \neq \emtpyset$.
  
  Dual: Dadas $a$,$b$,$c$,$d$ rectas distintas tales que $a \cap b$ y $c \cap d$
  existen, entonces $a \cap d$ y $b \cap c$ existen.\\

  Los 3 puntos diagonales de un cuadrángulo completo no son colineales.\\
  Dual: Las 3 rectas diagonales de un cuadrilátero completo no son concurrentes.\\
%  \begin{proof}[Prueba]
    Sea el cuadrilátero completo $a$,$b$,$c$,$d$.\\
    Observación:
    %\begin{multicols}{2}
      $a \cap b$ = \{$E$\}\\
      $a \cap c$ = \{$F$\}\\
      $a \cap d$ = \{$G$\}\\
      $c \cap d$ = \{$H$\}\\
      $b \cap d$ = \{$I$\}\\
      $b \cap c$ = \{$J$\}
    %\end{multicols}

   % \begin{flushleft}
    entonces $\overline{EH}$, $\overline{FI}$, $\overline{GJ}$ son las
    diagonales del cuadrángulo completo $a$, $b$, $c$, $d$.\\

    Por demostrar,\\
    %\begin{center}
      $\overline{EH} \cap \overline{FI} \cap \overline{GJ}$ = $\emptyset$

    %\begin{flushleft}
    %Consideremos el cuadrángulo completo $\msquare FGIJ$ las rectas que
    determinan son
 %   \begin{center}
      %\begin{multicols}{2}
        $\overline{FG}$ = $a$\\
        $\overline{IJ}$ = $b$\\
        $\overline{FI}$ = $p$\\
        $\overline{GJ}$ = $q$\\
        $\overline{FJ}$ = $c$\\
        $\overline{GI}$ = $d$\\
      %\end{multicols}

 %   \begin{center}
    $a \cap b$ = \{$E$\}
    $p \cap q$ = \{$R$\}
    $c \cap d$ = \{$H$\}

%    \begin{flushleft}
      Observación:\\
   %   \begin{center}
        $\overline{FI} \cap \overline{GJ}$ = $R$

%    \begin{flushleft}
    Pero $R \notin \overline{EH}$ por axioma 7.
  %\end{proof}

  %\begin{flushleft}
  Ej:\\
  Sean
%  \begin{center}
    \{$A$,$B$,$C$\} $\subseteq l$\\
    \{$A'$,$B'$,$C'$\} $\subseteq l$\\

  %  \begin{flushleft}
      donde los puntos son distintos y las rectas coplanares.\\

      ¿Cuántas proyectividades existen tales que

    \begin{center}
      $ABCX$ proyectividad $A'B'C'X'$
    \end{center}
        
 %   \begin{flushleft}
      para toda $x \in l$?\\

      Sea $l \neq l'$ y son coplanares, entonces\\

%  \begin{center}
    $l \cap l'$ = \{$P$\}\\
    $P \notin$ \{$A$,$B$,$C$,$A'$,$B'$,$C'$\}\\
  %\begin{flushleft}
    Puesto que $A \neq B$ y \{$A$,$B$\} $\subseteq l$\\
    entonces $l = \overline{AB}$.\\
    Análogamente, $l' = \overline{A'B'}$.\\

    Así,
 % \begin{center}
    $\overline{AB} \cap \overline{A'B'}$ = $l \cap l'$ = \{$P$\}\\
    entonces $\overline{AB'} \cap \overline{A'B}$ = \{$R$\} $\neq \emptyset$

%  \begin{flushleft}
    De igual manera,
  %\begin{center}
    $\overline{A'C} \cap \overline{AC'}$ = \{$Q$\}\\
    $R \neq Q$ entonces existe $\overline{RQ}$\\

 % \begin{flushleft}
    pues si,\\
  %\begin{center}
    $R=Q$\\
    $R \in \overline{AB'} \Rightarrow B' \in \overline{AR}$\\
    $Q \in \overline{AC'} \Rightarrow C' \in \overline{AQ}$\\
    $R \in \overline{A'B} \Rightarrow B \in \overline{A'R}$\\
    $Q \in \overline{A'C} \Rightarrow C \in \overline{A'Q}$\\

    $B' \neq C'$\\
    \{$B'$,$C'$\} $\subseteq \overline{AR} = \overline{AQ}$ pues $R=Q$\\
    $\Rightarrow l' = \overline{AR} = \overline{AQ}$\\
    $\Rightarrow A \in l'$\\

    $B \neq C$\\
    \{$B$,$C$\} $\subseteq l$\\
    \{$B$,$C$\} $\overline{A'R} = \overline{A'Q}$ pues $R=Q$\\
    $\Rightarrow l = \overline{A'R} = \overline{A'Q}$\\
    %\begin{flushleft}
      por lo tanto\\
    %\begin{center}
      $A' \in l$\\
 % \begin{flushleft}
    lo cual es una contradicción, pues,\\
%  \begin{center}
    $A \in l$ y $A \in l'$\\
    $\Rightarrow A = P$\\
    y $A' \in l'$ y $A' \in l$\\
    $\Rightarrow A' = P$\\
    por lo tanto\\
    $A = A'$ lo cual no es posible.
    

  

\printindex  

\end{document}