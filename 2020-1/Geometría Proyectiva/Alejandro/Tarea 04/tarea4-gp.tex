
  
\documentclass[12pts]{report}


\usepackage{amssymb}
\usepackage{amsmath}
\usepackage{amscd}
\usepackage{amsthm}
\usepackage[utf8]{inputenc}
\usepackage[spanish,mexico]{babel}
\usepackage{enumerate}
\usepackage[usenames]{color}


\usepackage{pgf,tikz}
\usetikzlibrary{arrows}

\usepackage[colorlinks=true, linkcolor=blue, urlcolor=red,
citecolor=green]{hyperref}

\voffset=-2cm
\hoffset=-2cm
\textwidth = 18cm
\textheight= 23 cm

\usepackage{iwona}
\usepackage{fancyhdr}
\pagestyle{fancy}
\fancyhf{}
\fancyhead[RE,LO]{\bfseries{Geometría Proyectiva}}
\fancyhead[LE,RO]{\bfseries{2020-1}}
\fancyfoot[RE,RO]{\bfseries{Noviembre 2018}}
\fancyfoot[LE,LO]{\bfseries{Tarea 04}}

\newcommand{\R}{\mathbb R}
\newcommand{\Q}{\mathbb Q}
\newcommand{\E}{\mathbb E}
\newcommand{\s}{\mathbb S}
\newcommand{\C}{\mathbb C}
\newcommand{\F}{\mathbb F}
\newcommand{\T}{\mathbb T}
\newcommand{\p}{\mathbb P}
\newcommand{\I}{\mathbb I}
\newcommand{\A}{\mathbb A}
\newcommand{\h}{\mathbb H}

\begin{document}
\begin{center}
\textcolor{blue}{\textbf{\large Guia de ejercicios para la Evaluación Parcial 04 }}\\
\vspace{0.5 cm}
\end{center}

%\textbf{\large Instrucciones:}
%\begin{itemize}
%\item Se sugiere trabajar en equipo los ejercicios y expresar cualquier duda en clase o por correo electrónico.
%\end{itemize}

\begin{center}
%\textcolor{red}{\textbf{\large EXAMEN PARCIAL 02\\ VIERNES
%13-SEPTIEMBRE-2019\\ De 19:00 a 21:00 HORAS - Salón P-213}}
\end{center}

\begin{enumerate}
\item Sean $\{A,B,C,D,E,F\}\subset l$. y las involuciones $\varphi: l\to l$ tal que $(AB)(DE)$, $\psi: l\to l$ tal que $(BC)(EF)$ y $\lambda: l\to l$ tal que $(CD)(FA)$. Demostrar que si dos de estas involuciones tiene un par de puntos en común entonces las tres tienen un par en común

\item Sean $\{A,M,N\}\subset l$. Demostrar que la proyectividad $MNA\overline{\wedge} MNA'$ es el producto de las involuciones $(AB)(MN)$ y $(A'B)(MN)$ donde $B$ es un punto arbitriario en $l$.

\item Sean $\{A,A',B,B'\}\subset l$. Demostrar que la involucion $(AA')(BB')$ puede ser expresado como el producto de $(AB)(A'B')$ y $(AB')(BA')$.

\item Sean $\{A,A',A'',A'''\}\subset l$. Demostrar que cualquier proyectividad que no es una involución puede ser expresado como el producto de $(AA'')(A'A')$ y $(AA''')(A'A'')$.

\item Sean $\{A,B,C,D\}\subset l$. Demostrar que si existe la proyectividad $ABCD \overline{\wedge} BACD$ entonces $H(A,B; C,D)$.

\item Sean $\{A,A',B,B',M,N\}\subset l$. Demostrar que si $H(A,B; M,N)$ y $H(A',B';M,N)$ entonces $M,N$ es un par de la involución $(AA')(BB')$.

\item Sean $\{A,B,C,D,E,F, P,Q\}\subset l$. Demostrar que si $(AD)(BE)(CF)$ y $A',B',C',D',E', F'$ son los conjugado armonico de $A,B,C,D,E,F$ con respecto a $P, Q$  respectivamente entonces $(A'D')(B'E')(C'F')$.

\item Sean $\{A,B,C,D,E,F\}\subset l$. Demostrar que si $ABCD \overline{\wedge} ABDE$ y $H(C,E;D,F)$ entonces $H(A,B;D,F)$.

\item Sean $\{A,B,C,D,E,F\}\subset l$. Demostrar que si $H(B,C;AD)$, $H(C,A;B,E)$ y $H(A,B;C,F)$ entonces $(AD)(BE)(CF)$.
\end{enumerate}
\end{document}