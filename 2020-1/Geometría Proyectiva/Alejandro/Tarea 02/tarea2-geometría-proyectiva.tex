
  
\documentclass[12pts]{report}


\usepackage{amssymb}
\usepackage{amsmath}
\usepackage{amscd}
\usepackage{amsthm}
\usepackage[utf8]{inputenc}
\usepackage[spanish,mexico]{babel}
\usepackage{enumerate}
\usepackage[usenames]{color}


\usepackage{pgf,tikz}
\usetikzlibrary{arrows}

\usepackage[colorlinks=true, linkcolor=blue, urlcolor=red,
citecolor=green]{hyperref}

\voffset=-2cm
\hoffset=-2cm
\textwidth = 18cm
\textheight= 23 cm

\usepackage{iwona}
\usepackage{fancyhdr}
\pagestyle{fancy}
\fancyhf{}
\fancyhead[RE,LO]{\bfseries{Geometría Proyectiva}}
\fancyhead[LE,RO]{\bfseries{2020-1}}
\fancyfoot[RE,RO]{\bfseries{Septiembre 2018}}
\fancyfoot[LE,LO]{\bfseries{Tarea 02}}

\newcommand{\R}{\mathbb R}
\newcommand{\Q}{\mathbb Q}
\newcommand{\E}{\mathbb E}
\newcommand{\s}{\mathbb S}
\newcommand{\C}{\mathbb C}
\newcommand{\F}{\mathbb F}
\newcommand{\T}{\mathbb T}
\newcommand{\p}{\mathbb P}
\newcommand{\I}{\mathbb I}
\newcommand{\A}{\mathbb A}
\newcommand{\h}{\mathbb H}

\begin{document}
\begin{center}
\textcolor{blue}{\textbf{\large Guia de ejercicios para la Evaluación Parcial 02 }}\\
\vspace{0.5 cm}
\end{center}

%\textbf{\large Instrucciones:}
%\begin{itemize}
%\item Se sugiere trabajar en equipo los ejercicios y expresar cualquier duda en clase o por correo electrónico.
%\end{itemize}

\begin{center}
%\textcolor{red}{\textbf{\large EXAMEN PARCIAL 02\\ VIERNES
%13-SEPTIEMBRE-2019\\ De 19:00 a 21:00 HORAS - Salón P-213}}
\end{center}

\begin{enumerate}
\item Sea $l$ una recta y $\{A,B,C,D\}\subset l$. Demostrar que si $H(A,B;C,D)$ entonces $H(D,C; B,A)$

\item Sea $L$ un punto y $a,b,c,d$ cuatro rectas distintas tales que $a\cap b\cap c\cap d =\{L\}$. Demostrar que si $H(a,b;c,d)$ entonces $H(d,c; b,a)$

\item Sean $\triangle PQR$ , $\{A,A'\}\subset \overline{QR}$ y $\{B, B'\}\subset \overline{PR}$. Demostrar que si $H(A,A';Q,R)$ y $H(B,B';R,P)$ entonces $H(P,Q;C,C')$ donde $\{C\}=\overline{AB'}\cap\overline{BA'}$ y $\{C'\}=\overline{AB}\cap\overline{A'B'}$.

\item Sean $\triangle pqr$ , $a\cap a'=\{ q\cap r\}$ y $b\cap b'=\{p\cap r\}$. Demostrar que si $H(a,a';q,r)$ y $H(b,b';r,p)$ entonces $H(p,q;c,c')$ donde $c=\overline{(a\cap b')(a'\cap b)}$ y $c'=\overline{(a\cap b')(a'\cap b)}$.


\item Sean $l$ una recta y $\{A,B,C\}\subset l$ distintos. Construir tres proyectividades  $\lambda, \varphi, \psi $ tales que $\sigma=\psi\circ\varphi\circ\lambda$ cumpla que:
$$ABC\;\overset{\sigma}{\overline{\wedge}}\; BCA$$

\item Sean $L$ un punto y $a, b , c$ rectas distintas tales que $a\cap b\cap c = \{L\}$. Construir tres proyectividades  $\lambda, \varphi, \psi $ tales que $\sigma=\psi\circ\varphi\circ\lambda$ cumpla que:
$$abc\;\overset{\sigma}{\overline{\wedge}}\; bca$$

\item Sean $l$ una recta y $\{A,B,C,D\}\subset l$ distintos. Demostrar que existen $\varphi, \psi, \lambda$ proyectividades tales que:
$$ABCD\; \overset{\lambda}{\overline{\wedge}}\; BADC\; \overset{\varphi}{\overline{\wedge}}\; CDAB \; \overset{\psi}{\overline{\wedge}}\;DCBA$$

\item Sean $L$ un punto y $a,b ,c ,d$ rectas distintas tales que $a\cap b\cap c\cap d=\{L\}$. Demostrar que existen $\varphi, \psi, \lambda$ proyectividades tales que:
$$abcd\; \overset{\lambda}{\overline{\wedge}}\; badc\; \overset{\varphi}{\overline{\wedge}}\; cdab \; \overset{\psi}{\overline{\wedge}}\;dcba$$

\item Dualizar el teorema de pappus y demostrarlo.

\item Sean $H(a,b;c,d)$ y $H(a',b';c',d)$. Demostrar que existe una única proyectividad $\varphi$ tal que:
$$abcd\;\overset{\varphi}{\overline{\wedge}}\; a'b'c'd'$$



\end{enumerate}
\end{document}