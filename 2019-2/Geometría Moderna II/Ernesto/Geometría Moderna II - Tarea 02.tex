\documentclass[12pt]{report}

\usepackage{amssymb}
\usepackage{amsmath}
\usepackage{amscd}
\usepackage{amsthm}
\usepackage[utf8]{inputenc}
\usepackage[spanish,mexico]{babel}
\usepackage{enumerate}
\usepackage[usenames]{color}
\numberwithin{section}{chapter}

\usepackage{pgf,tikz}
\usetikzlibrary{arrows}

\usepackage[colorlinks=true, linkcolor=blue, urlcolor=red,
citecolor=green]{hyperref}

\voffset=-2cm
\hoffset=-3cm
\textwidth = 19.5cm
\textheight= 23 cm

\usepackage{iwona}
\usepackage{fancyhdr}
\pagestyle{fancy}
\fancyhf{}
\fancyhead[RE,LO]{\bfseries{Geometría Moderna II}}
\fancyhead[LE,RO]{\bfseries{2019-2}}
\fancyfoot[RE,RO]{\bfseries{Marzo 2019}}
\fancyfoot[LE,LO]{\bfseries{Tarea 02}}

\newcommand{\R}{\mathbb R}
\newcommand{\Q}{\mathbb Q}
\newcommand{\E}{\mathbb E}
\newcommand{\s}{\mathbb S}
\newcommand{\C}{\mathbb C}
\newcommand{\F}{\mathbb F}
\newcommand{\T}{\mathbb T}
\newcommand{\p}{\mathbb P}
\newcommand{\I}{\mathbb I}
\newcommand{\A}{\mathbb A}


\begin{document}
\begin{center}
\textcolor{blue}{\textbf{\large Guía de ejercicios para al Evaluación Parcial 02}}\\
\vspace{0.5 cm}
\textcolor{red}{\textbf{\large EXAMEN PARCIAL 02 \\ VIERNES
29-MARZO-2019\\ De 19:00 a 21:00 HORAS - Salón P-108}}
\end{center}


\begin{center}
\textcolor{purple}{\textbf{\large Inversión}}
\end{center}

\begin{enumerate}

\item Sean $\zeta(A, \alpha)$ y $P \neq A$. Demostrar que si $I_{\zeta(A, \alpha)} (P) = P'$ y $\zeta(A, \alpha) \cap \overline{PP'} = \{ R, S \}$ entonces $\overline{PP'}\{P, P'; R, S\} = -1$.

\item Demostrar que si $\{A,B,C,D,E\} \subseteq l$ (conjunto de puntos distintos) tales que $l\{A,B;C,D\}=-1$ entonces para cualquier $\eta \in \R^+$ se tiene que $l\{I_{\zeta(E,\eta)}(A), I_{\zeta(E,\eta)}(B);I_{\zeta(E,\eta)}(C),I_{\zeta(E,\eta)}(D)\}=-1$.

\item Demostrar que si $\Gamma$ es una familia de circunferencias coaxiales que tiene un par de puntos límite $\{L,L'\}$ entonces para cualquier $\zeta(P,\rho) \in \Gamma$ se tiene que $I_{\zeta(P, \rho)}(L) = L'$.

\item Sea $\{A, P, Q\}$ un conjunto de puntos en posición general. Demostrar que si $I_{\zeta(A, \alpha)} (P) = P'$ y \break $I_{\zeta(A, \alpha)} (Q) = Q'$ entonces $\{P, P', Q, Q'\}$ es un conjunto concíclico de puntos y la circunferencia que los contiene es ortogonal a $\zeta(A, \alpha)$.

\item Si una circunferencia es invertida en una circunferencia, ¿el centro de la primera es invertido en la segunda?

\item Sea $\Gamma$ la familia de circunferencias coaxiales a las que pertenece $\zeta(A, \alpha)$ y $\zeta(B, \beta)$ con $A \neq B$. Demostrar que si $I_{\zeta(P, \rho)} [\zeta(A, \alpha)] = \zeta(B, \beta)$ entonces $\zeta(P, \rho) \in \Gamma$.

\item Sean $\zeta (A,\alpha)$ y $\zeta(B,\beta)$ con $A \neq B$. Construir $\zeta(P, \rho)$ tal que $$I_{\zeta(P, \rho)} [\zeta(A, \alpha)] = \zeta(C, \gamma) \qquad ; \qquad I_{\zeta(P, \rho)} [\zeta(B, \beta)] = \zeta(D, \gamma)$$

\item Sean $\zeta(A,\alpha)$ y $X\neq A$. Demostrar que para cualquier $\zeta(P,\rho)$ si $I_{\zeta(P,\rho)}[\zeta(A,\alpha)] = \zeta(B,\beta)$ entonces $I_{\zeta(B,\beta)}(I_{\zeta(P,\rho)}(X)) = I_{\zeta(P,\rho)}(I_{\zeta(A,\alpha)}(X))$.

%\item Demostrar que si $\zeta(A,\alpha)$ y $\zeta(B, \beta)$ son ortogonales entonces $I_{\zeta(A, \alpha)}(B)$ es el punto medio del segmento determinado por los puntos de $\zeta(A,\alpha) \cap \zeta(B,\beta)$.

\item Demostrar que si $\zeta (A,\alpha)$ y $\zeta(B,\beta)$ con $A \neq B$ se intersecan en dos puntos entonces sus circunferencias de antisimilitud son ortogonales.

%\item Sean $\{A, B, C\}$ un conjunto de puntos en posición general y se considérese $\zeta (A,\alpha)$, $\zeta(B,\beta)$ y $\zeta(C, \gamma)$. Construir $\zeta(P,\rho)$ tal que las imágenes de estas circunferencias bajo $I_{\zeta(P,\rho)}$ sean tres circunferencias con centros colineales.

\item Sea $\zeta(A, \alpha)$ y $P\neq A$. Demostrar que si $I_{\zeta(A, \alpha)}(P)=P'$ entonces para cualquier $X \in \zeta(A, \alpha)$ se tiene que $\frac{XP}{XP'} = k$ para alguna $k \in \R$.

\end{enumerate}

\end{document}
