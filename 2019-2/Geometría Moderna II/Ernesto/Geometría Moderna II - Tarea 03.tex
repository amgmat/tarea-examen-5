\documentclass[12pt]{report}

\usepackage{amssymb}
\usepackage{amsmath}
\usepackage{amscd}
\usepackage{amsthm}
\usepackage[utf8]{inputenc}
\usepackage[spanish,mexico]{babel}
\usepackage{enumerate}
\usepackage[usenames]{color}
\numberwithin{section}{chapter}

\usepackage{pgf,tikz}
\usetikzlibrary{arrows}

\usepackage[colorlinks=true, linkcolor=blue, urlcolor=red,
citecolor=green]{hyperref}

\voffset=-2cm
\hoffset=-3cm
\textwidth = 19.5cm
\textheight= 23 cm

\usepackage{iwona}
\usepackage{fancyhdr}
\pagestyle{fancy}
\fancyhf{}
\fancyhead[RE,LO]{\bfseries{Geometría Moderna II}}
\fancyhead[LE,RO]{\bfseries{2019-2}}
\fancyfoot[RE,RO]{\bfseries{Marzo 2019}}
\fancyfoot[LE,LO]{\bfseries{Tarea 03}}

\newcommand{\R}{\mathbb R}
\newcommand{\Q}{\mathbb Q}
\newcommand{\E}{\mathbb E}
\newcommand{\s}{\mathbb S}
\newcommand{\C}{\mathbb C}
\newcommand{\F}{\mathbb F}
\newcommand{\T}{\mathbb T}
\newcommand{\p}{\mathbb P}
\newcommand{\I}{\mathbb I}
\newcommand{\A}{\mathbb A}


\begin{document}
\begin{center}
\textcolor{blue}{\textbf{\large Guía de ejercicios para al Evaluación Parcial 03}}\\
\vspace{0.5 cm}
\textcolor{red}{\textbf{\large EVALUACIÓN PARCIAL 03 \\ DEL LUNES 08-ABRIL-2019 AL VIERNES
12-ABRIL-2019\\ De 20:00 a 21:00 HORAS - Salón P-108}}
\end{center}

\textbf{INSTRUCCIONES}:
\begin{itemize}
\item La tercera evaluación parcial consistirá en una exposición de un problema de la siguiente lista.
\item La asistencia a las sesiones de exposiciones forma parte de dicha evaluación.
\item Cada estudiante deberá exponer solamente algunos ejercicios de la presente lista. La asignación de ejercicios se realizará en el momento que inicie la sesión de exposiciones. \textbf{Sugerencia: Recomendamos ampliamente resolver TODA la tarea con anticipación.}
\item Deberán entregar por escrito los ejercicios que se expongan a más tardar el \textbf{Viernes 12 de abril de 2019 a las 20:00 horas}.
\end{itemize}

\begin{center}
\textcolor{purple}{\textbf{\large Polos y polares}}
\end{center}

\begin{enumerate}

\item Demostrar que en un triángulo rectángulo las mediatrices concurren en el punto medio de la hipotenusa.

\item Sea $\zeta(P, \rho)$ y $L$ un punto en el plano tal que $\rho < |LP|$. Demostrar que la recta que contiene a los puntos de tangencia a $\zeta(P,\rho)$ de las tangentes a $\zeta(P, \rho)$ desde $L$ es la polar de $L$. 

\item Sea $l$ una recta y $L$ un punto en el plano. Construir $\zeta(P, \rho)$ tal que $l$ sea polar de
$L$ respecto a $\zeta(P, \rho)$. ¿Cuántas maneras hay de hacerlo?

\item Demostrar que
\begin{enumerate}
\item Un par de puntos son conjugados con respecto a $\zeta(P, \rho)$ entonces sus polares son rectas conjugadas respecto a $\zeta(P, \rho)$.
\item Un par de rectas son conjugadas con respecto a $\zeta(P, \rho)$ entonces sus polos son puntos conjugados respecto a $\zeta(P, \rho)$.
\end{enumerate}

\item Demostrar que si $a$ y $b$ son rectas conjugadas respecto a $\zeta(P, \rho)$ tales que $a \cap b = \{X\}$ con $\rho < |XP|$ entonces $|a \cap \zeta(P, \rho)| = 2$ y $|b \cap \zeta(P, \rho)| = 0$ o $|a \cap \zeta(P, \rho)| = 0$ y $|b \cap \zeta(P, \rho)| = 2$.

\item Sea $\mathbb L$ el conjunto de rectas incidentes en el punto $L$ en el plano. Determinar el lugar geométrico de los puntos conjugados a $L$ en $l$ para cada $l \in \mathbb L$.

\item Considerar $\square ABCD$ un cuadrado y $\zeta(P, \rho)$ para demostrar que $A$ es conjugado de $C$ respecto a $\zeta(P, \rho)$ si y solamente si $B$ es conjugado de $D$ respecto a $\zeta(P, \rho)$.

\item Sean $\zeta(A, \alpha)$ y $ \zeta(B, \beta)$ circunferencias con la propiedad de tener a la recta $t$ como una tangente común. Demostrar que si $\Gamma$ es la familia de circunferencias coaxiales a la que pertenecen $\zeta(A, \alpha)$ y $ \zeta(B, \beta)$, $\zeta(A, \alpha) \cap t = \{P\}$ y $\zeta(B, \beta) \cap t = \{Q\}$ entonces $P$ y $Q$ son puntos conjugados con respecto a $\zeta(X, \xi)$ para cualquier $\zeta(X,\xi) \in \Gamma$.

\item Sea $\zeta(P,\rho)$ y $A \neq P$. Construir la polar de $A$ con el uso de únicamente regla.

\item Sea $\zeta(P,\rho)$ y $A$ un punto en el plano tal que $\rho \leq |PA|$. Construir las tangentes a $\zeta(P, \rho)$ por $A$ con el uso de únicamente regla.

\item Triángulo autopolar\footnote{$\triangle ABC$ es \textbf{autopolar respecto a $\zeta(P, \rho)$} si y solamente si cada vértice es polo del lado opuesto respecto a $\zeta(P, \rho)$.}:
\begin{enumerate}
\item Construir un triángulo que sea autopolar respecto a $\zeta(P,\rho)$ dada una recta $c$ y $A \in c$.
\item Demostrar que el otrocentro de un triángulo autopolar respecto a $\zeta(P,\rho)$ es $P$.
\end{enumerate}

\item Demostrar que si un triángulo es autopolar respecto a $\zeta(P,\rho)$ entonces:
\begin{enumerate}
\item Solamente un vértice del triángulo se encuentra dentro de la circunferencia.
\item El ángulo interno del vértice que se encuentra dentro de la circunferencia es mayor a uno recto.
\end{enumerate}

\item Dado $\triangle ABC$ con la propiedad de tener un ángulo interno mayor que uno recto. Construir una circunferencia $\zeta(P,\rho)$ tal que $\triangle ABC$ sea autopolar respecto a $\zeta(P,\rho)$. ¿Cuántas maneras hay de hacerlo? 

\item Demostrar que si $\{A,B,C,D\} \subseteq \zeta(P,\rho)$ entonces el triángulo diagonal de $\square ABCD$ es autopolar respecto a $\zeta(P,\rho)$.

\item Demostrar que si $\triangle ABC$ es autopolar respecto a $\zeta(P,\rho)$ entonces el inverso de la circunferencia que inscribe a $\triangle ABC$ respecto a $\zeta(P, \rho)$ es la circunferencia de los nueve puntos de $\triangle ABC$.

\end{enumerate}

\end{document}
