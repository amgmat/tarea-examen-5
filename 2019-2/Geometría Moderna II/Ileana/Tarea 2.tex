\documentclass[12pt]{article}

\usepackage[T1]{fontenc}
\usepackage{lmodern}

\usepackage{amssymb}
\usepackage{amsmath}
\usepackage{amscd}
\usepackage{amsthm}
\usepackage[latin1]{inputenc}
\usepackage[spanish,mexico]{babel}
\usepackage{enumerate}
\usepackage{pgf,tikz}
\usepackage{makeidx}
\usetikzlibrary{arrows}
\usepackage{multicol}

\usepackage[colorlinks=true, linkcolor=blue, urlcolor=blue, citecolor=red]{hyperref}

\usepackage{graphicx}
\usepackage{wrapfig}

\voffset=-2 cm
\hoffset=-3 cm
\textwidth=20 cm
\textheight=22 cm


\theoremstyle{definition}
\newtheorem{teo}{\textcolor{red}{Teorema}}[section]
\newtheorem{df}[teo]{\textcolor{cyan}{Definici�n}}
\newtheorem{ej}[teo]{\textcolor{red}{Ejercicio}}
\newtheorem{prop}[teo]{\textcolor{red}{Proposici�n}}
\newtheorem{lema}[teo]{\textcolor{red}{Lema}}
\newtheorem{cor}[teo]{\textcolor{red}{Corolario}}
\newtheorem{obs}[teo]{\textcolor{purple}{Observaci�n}}
\newtheorem{ejp}[teo]{\textcolor{purple}{Ejemplo}}

\newenvironment{pba}{\noindent\textbf{\textcolor{blue}{Prueba:}}}{\begin{flushright}
$\square$ \end{flushright}}
\newenvironment{dem}{\noindent\textbf{\textcolor{blue}{Demostraci�n}:}}{\begin{flushright}
\rule{1ex}{1ex} \end{flushright}}

\newcommand{\N}{\mathbb N}
\newcommand{\Z}{\mathbb Z}
\newcommand{\Q}{\mathbb Q}
\newcommand{\R}{\mathbb R}



\title{Tarea 2 \\
Geometr�a Moderna II}


\begin{document}

\maketitle

\begin{enumerate}

\item Sean $A$ y $B$ dos puntos en el plano tales que $A \neq B$, $\zeta(A, \alpha)$, $\zeta(B, \beta)$ tales que $\zeta(A, \alpha) \cap \zeta(B, \beta) = \emptyset$ y una no contenga a la otra. Tambi�n sea $l$ una recta incidente en $A$ y $m$ la recta paralela a $l$ por $B$; $l \cap \zeta(A, \alpha) = \left\lbrace P, Q\right\rbrace$, $m \cap \zeta(B, \beta) = \{ R, S\}$. 

Demostrar: 1) $\overline{PR} \cap \overline{SQ} \cap \overline{AB} \neq \emptyset$. 

2) 1) $\overline{PS} \cap \overline{QR} \cap \overline{AB} \neq \emptyset$ 

3) $n$ es tangente a $\zeta(A, \alpha)$ por alguno de los puntos anteriores $\Leftrightarrow \, n$ es tangente a $\zeta(B, \beta).$

\vspace{0.5cm} 

\item Si $I_{\zeta(P, \rho)} [\zeta(A, \alpha)] = \zeta(B, \beta)$, entonces $\{ \zeta(P, \rho), \zeta(A, \alpha), \zeta(B, \beta) \} \subset \Gamma$, donde $\Gamma$ es la familia de circunferencias generada por $\zeta(A, \alpha), \zeta(B, \beta)$

\vspace{0.5cm} 

\item Si tenemos $\zeta(A, \alpha), \zeta(B, \beta)$ tales que una est� contenida en la otra, entonces �qu� centro de similitud funciona como centro de $\zeta(P, \rho)$?, donde $I_{\zeta(P, \rho)} [\zeta(A, \alpha)] = \zeta(B, \beta)$

\vspace{0.5cm} 

\item Encontrar $\zeta(P, \rho)$ tal que: $I_{\zeta(P, \rho)} [\zeta(A, \alpha)] = \zeta(C, \gamma)$ y adem�s $I_{\zeta(P, \rho)} [\zeta(B, \beta)] = \zeta(C, \gamma)$

\vspace{0.5cm} 

\item Sean $\zeta(A, \alpha)$ y P un punto en el plano.

Si $I_{\zeta(A, \alpha)} \, (P) = P'$ y $\zeta(A, \alpha) \cap \overline{PP'} = \{ R, S \}$, entonces $\overline{PP'}\{P, P'; R, S\} = -1$ 
 
\vspace{0.5cm} 

\item Si una circunferencia es invertida en una circunferencia, �el centro de la primera es invertido en la segunda?

\vspace{0.5cm} 

\item Sean $\{A, P, Q\}$ un conjunto de puntos en posici�n general, $\zeta(A, \alpha)$, $I_{\zeta(A, \alpha)} \, (P) = P'$ y $I_{\zeta(A, \alpha)} \, (Q) = Q'$. Demostrar que:

1) $\triangle APQ \cong \triangle AP'Q'$.

2) $\{P, P', Q, Q'\} \in \zeta(D, \delta)$, donde $\zeta(D, \delta)$ es ortogonal a $\zeta(A, \alpha).$

\end{enumerate}
\end{document}