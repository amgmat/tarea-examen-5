\documentclass[12pt]{article}

\usepackage[T1]{fontenc}
\usepackage{lmodern}

\usepackage{amssymb}
\usepackage{amsmath}
\usepackage{amscd}
\usepackage{amsthm}
\usepackage[latin1]{inputenc}
\usepackage[spanish,mexico]{babel}
\usepackage{enumerate}
\usepackage{pgf,tikz}
\usepackage{makeidx}
\usetikzlibrary{arrows}
\usepackage{multicol}

\usepackage[colorlinks=true, linkcolor=blue, urlcolor=blue, citecolor=red]{hyperref}

\usepackage{graphicx}
\usepackage{wrapfig}

\voffset=-2 cm
\hoffset=-3 cm
\textwidth=20 cm
\textheight=22 cm


\theoremstyle{definition}
\newtheorem{teo}{\textcolor{red}{Teorema}}[section]
\newtheorem{df}[teo]{\textcolor{cyan}{Definici�n}}
\newtheorem{ej}[teo]{\textcolor{red}{Ejercicio}}
\newtheorem{prop}[teo]{\textcolor{red}{Proposici�n}}
\newtheorem{lema}[teo]{\textcolor{red}{Lema}}
\newtheorem{cor}[teo]{\textcolor{red}{Corolario}}
\newtheorem{obs}[teo]{\textcolor{purple}{Observaci�n}}
\newtheorem{ejp}[teo]{\textcolor{purple}{Ejemplo}}

\newenvironment{pba}{\noindent\textbf{\textcolor{blue}{Prueba:}}}{\begin{flushright}
$\square$ \end{flushright}}
\newenvironment{dem}{\noindent\textbf{\textcolor{blue}{Demostraci�n}:}}{\begin{flushright}
\rule{1ex}{1ex} \end{flushright}}

\newcommand{\N}{\mathbb N}
\newcommand{\Z}{\mathbb Z}
\newcommand{\Q}{\mathbb Q}
\newcommand{\R}{\mathbb R}



\title{Tarea 2 \\
Geometr�a Moderna II}


\begin{document}

\maketitle

\begin{enumerate}

\item En un tri�ngulo rect�ngulo, las mediatrices concurren en el punto medio de la hipotenusa. 

\item Sea $\zeta(O, r)$. Dado $L$ punto en el plano tal que $|LO| > r$, �por qu� se puede construir el polo trazando las tangentes a $\zeta(O, r)$ por $L$? �Cu�l es el polo? 

\item  Sea $A$ punto en el plano. Encontrar la polar de $A$ usando �nicamente regla.

\item Sea $\zeta(O, r)$ y $A$ un punto en el plano tal que $|OA| > r$. Encontrar las tangentes a $\zeta(O, r)$ por $A$ usando �nicamente regla.

\item Sea $\zeta(O, r)$ y $A$ un punto en el plano tal que $|OA| = r$. Encontrar la tangente a $\zeta(O, r)$ por $A$ usando �nicamente regla.

\item Sean $\zeta(O, r)$ y $a, b$ dos rectas conjugadas tales que $a \cap b = \{X\}$ y $|XO| > r$. Demostrar que $|a \cap \zeta(O, r)| = 2$ y $|b \cap \zeta(O, r)| = 0$.

\item Encontrar el lugar geom�trico de un punto cuyas polares con respecto a dos circunferencias dadas forman un �ngulo fijo entre ellas.

\item Sea $ABCD$ un cuadrado y $\zeta(O, r)$. 
Si $A$ es conjugado de $C$ respecto a $\zeta(O, r)$, entonces $B$ es conjugado de $D$ respecto a $\zeta(O, r)$.

\item Sea $l$ una recta y $L$ un punto en el plano. Encontrar $\zeta(O, r)$ tales que $l$ sea polar de $L$ respecto a $\zeta(O, r)$. �Cu�ntas formas hay de hacerlo?

\item Sean $\zeta(A, \alpha), \zeta(B, \beta), \, t$ tangente com�n a ambas, $\zeta(A, \alpha) \cap t = \{P\}$ y $\zeta(A, \alpha) \cap t = \{Q\}$. Demostrar que $P$ y $Q$ son puntos conjugados con respecto a cualquier circunferencia que pertenezca a la misma familia que $\zeta(A, \alpha)$ y $\zeta(B, \beta).$ 

\vspace{0.5cm}

Se dice que un tri�ngulo es autopolar con respecto a una circunferencia cuando cada v�rtice es polo del lado opuesto.

\item Construir un tri�ngulo autopolar dados $A$ y $B$ puntos en el plano tales que $B \in a$. 

\item Sea $\triangle ABC$. Demostrar que el otrocentro del $\triangle ABC$ es el centro de la circunferencia que hace que $\triangle ABC$ sea autopolar.

\item Dada $\zeta(O, r)$, �cu�ntos tri�ngulos autopolares respecto a $\zeta(O, r)$ existen?

\item Dado $\triangle ABC$, �cu�ntas circunferencias existen respecto a las cuales $\triangle ABC$ es autopolar? 

\end{enumerate}
\end{document}