\documentclass[12pts]{report}


\usepackage{amssymb}
\usepackage{amsmath}
\usepackage{amscd}
\usepackage{amsthm}
\usepackage[utf8]{inputenc}
\usepackage[spanish,mexico]{babel}
\usepackage{enumerate}
\usepackage[usenames]{color}


\usepackage{pgf,tikz}
\usetikzlibrary{arrows}

\usepackage[colorlinks=true, linkcolor=blue, urlcolor=red,
citecolor=green]{hyperref}

\voffset=-2cm
\hoffset=-2cm
\textwidth = 18cm
\textheight= 23 cm

\usepackage{iwona}
\usepackage{fancyhdr}
\pagestyle{fancy}
\fancyhf{}
\fancyhead[RE,LO]{\bfseries{Seminario de Topologia B}}
\fancyhead[LE,RO]{\bfseries{2019-2}}
\fancyfoot[RE,RO]{\bfseries{Marzo 2019}}
\fancyfoot[LE,LO]{\bfseries{Tarea 02}}

\newcommand{\R}{\mathbb R}
\newcommand{\Q}{\mathbb Q}
\newcommand{\E}{\mathbb E}
\newcommand{\s}{\mathbb S}
\newcommand{\C}{\mathbb C}
\newcommand{\F}{\mathbb F}
\newcommand{\T}{\mathbb T}
\newcommand{\p}{\mathbb P}
\newcommand{\I}{\mathbb I}
\newcommand{\A}{\mathbb A}
\newcommand{\h}{\mathbb H}

\begin{document}
\begin{center}
\textcolor{blue}{\textbf{\large Guía de ejercicios para al Evaluación Parcial 02}}\\
\vspace{0.5 cm}
\textcolor{red}{\textbf{\large EXAMEN PARCIAL 02 \\ VIERNES
29-MARZO-2019\\ De 19:00 a 21:00 HORAS - Salón P-108}}
\end{center}

\begin{enumerate}
\item Sean $X$, $Y$ y $Z$ espacios topológicos. Consideremos $f,g: X\to Y$ y $\psi ,\varphi: Y \to Z$ funciones continuas entre estos espacios. Demostrar que:
\begin{enumerate}
\item Si $f\simeq g$ relativo a $A\subset X$ entonces $\varphi \circ f \simeq \varphi\circ g$ relativo a $A$.
\item Si $\varphi \simeq \phi$ relativo a $B\subset Y$ entonces $\varphi \circ f \simeq \psi \circ f$ relativo a $f^{-1}[B]$
\end{enumerate} 
\item Sean $X$, $Y$ espacios topológicos. Demostrar que si $Y$ es conectable por trayectorias entonces para cualquier $f,g : X\to Y$ funciones continuas nulhomotópicas tenemos que $f\simeq g$.
\item Demostrar que la relacion $\simeq$ es una relación de equivalencia en la clase de todos los espacios topológicos.
\item Sean $X$, $Y$ y $Z$ espacios topológicos. Entonces:
\begin{enumerate}
\item Demostrar que si $\varphi: (X,x) \to (Y,y)$, $\psi: (Y,y) \to (Z,z)$ son funciones basadas,
$$(\psi \circ \varphi)_* = \psi_*\circ \varphi_* : \pi(X,x)\to \pi(Z,z)$$
\item Demostrar que si $Id_X: (X,x)\to (X,x)$ es la identidad en X.
$$(Id_X)_* = Id_{\pi(X,x)}$$
\end{enumerate}
\item Demostrar que si $f,g :X \to \s^2$ son continuas tales que $\forall x\in X,\; f(x)\neq -g(x)$ entonces $f\simeq g$.
\item Sean $X$ un espacio topológico simplemente conexo y $\{x_0,x_1\}\subset X$. Demostar que si $f,g: I \to X$ son dos trayectorias con $f(0)=g(0)=x_0$ y $f(1)=g(1)=x_1$ entonces $f\simeq g$ relativo al $\{0,1\}$.
\item Sean $Y$ y $Z$ espacios topológicos y $p: Y \to Z$ una función cubriente. Demostrar que para todo $z\in Z$ el subespacio $p^{-1}(y)$ es un espacio discreto en Y.
\item Encontrar una función cubriente de $\s^1$ en $\s^1$ que no sea la función identidad.
\item Demostrar que $\R$ es un espacio cubriente del $\s^1$.
\item Demostrar que el toro es un espacio cubriente de la botella de Klein.


\end{enumerate}
\end{document}