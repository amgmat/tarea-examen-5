\documentclass[12pts]{report}


\usepackage{amssymb}
\usepackage{amsmath}
\usepackage{amscd}
\usepackage{amsthm}
\usepackage[utf8]{inputenc}
\usepackage[spanish,mexico]{babel}
\usepackage{enumerate}
\usepackage[usenames]{color}


\usepackage{pgf,tikz}
\usetikzlibrary{arrows}

\usepackage[colorlinks=true, linkcolor=blue, urlcolor=red,
citecolor=green]{hyperref}

\voffset=-2cm
\hoffset=-2cm
\textwidth = 18cm
\textheight= 23 cm

\usepackage{iwona}
\usepackage{fancyhdr}
\pagestyle{fancy}
\fancyhf{}
\fancyhead[RE,LO]{\bfseries{Seminario de Topologia B}}
\fancyhead[LE,RO]{\bfseries{2019-2}}
\fancyfoot[RE,RO]{\bfseries{Junio 2019}}
\fancyfoot[LE,LO]{\bfseries{Examen Final}}

\newcommand{\R}{\mathbb R}
\newcommand{\Q}{\mathbb Q}
\newcommand{\E}{\mathbb E}
\newcommand{\s}{\mathbb S}
\newcommand{\C}{\mathbb C}
\newcommand{\F}{\mathbb F}
\newcommand{\T}{\mathbb T}
\newcommand{\p}{\mathbb P}
\newcommand{\I}{\mathbb I}
\newcommand{\A}{\mathbb A}
\newcommand{\h}{\mathbb H}

\begin{document}
\begin{center}
\textbf{\LARGE {SEMINARIO DE TOPOLOGÍA B}}
\end{center}

\begin{center}
\textbf{{\large 2019-2 (07-JUNIO-2019)}}
\end{center}

\begin{center}
\textbf{{\large EXAMEN FINAL}}
\end{center}

{\bf INSTRUCCIONES:} Justificar y argumentar todos los resultados que se realicen. Resolver únicamente cinco ejercicios, de entregar más de cinco ejercicios se anulará el ejercicio de mayor puntaje.

\begin{enumerate}
\item Demostrar que todo nudo poligonal tiene una proyección regular.

\item Encontrar un nudo que no sea $3$-coloreable ni $5$-coloreable.

\item Demostrar que la suma conexa de dos nudos $3$-coloreables es un nudo $3$-coloreable.

\item Sean $(X,\tau_X)$ y $(Y,\tau_Y)$ espacios topológicos. Demostrar que si $Y$ es conectable por trayectorias entonces para cualesquiera $f: X \to Y$ y $g : X\to Y$ funciones continuas nulhomotópicas se tiene que $f\simeq g$.

\item Encontrar una función cubriente de $\s^1$ en $\s^1$ distinta de la función identidad.

\item Demostrar que si $K$ un nudo dócil y $D(k)$ un diagrama de $K$ con $n$ cruces entonces $D(k)$ determina $n+2$ regiones en el plano del diagrama.

\item Demostrar que 
$$\Delta_{p,q}(x)= |\Delta_{q,p}(x)|$$

\end{enumerate}
\end{document}