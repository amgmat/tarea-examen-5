\documentclass[12pt]{report}

\usepackage{amssymb}
\usepackage{amsmath}
\usepackage{amscd}
\usepackage{amsthm}
\usepackage[utf8]{inputenc}
\usepackage[spanish,mexico]{babel}
\usepackage{enumerate}
\usepackage[usenames]{color}
\numberwithin{section}{chapter}

\usepackage{pgf,tikz}
\usetikzlibrary{arrows}

\usepackage{multicol}

\usepackage{graphicx}
\usepackage{subfigure}

\usetikzlibrary{knots,hobby,decorations.pathreplacing,shapes.geometric,calc}
\tikzset{knot diagram/every strand/.append style={ultra thick,red}, show curve controls/.style={postaction=decorate, decoration={show path construction, curveto code={\draw [blue, dashed](\tikzinputsegmentfirst) -- (\tikzinputsegmentsupporta) node [at end, draw, solid, red, inner sep=2pt]{}; \draw [blue, dashed] (\tikzinputsegmentsupportb) -- (\tikzinputsegmentlast) node [at start, draw, solid, red, inner sep=2pt]{} node [at end, fill, blue, ellipse, inner sep=2pt]{};}}}, show curve endpoints/.style={ postaction=decorate, decoration={show path construction, curveto code={\node [fill, blue, ellipse, inner sep=2pt] at (\tikzinputsegmentlast) {};}}}}



\usepackage[colorlinks=true, linkcolor=blue, urlcolor=red,
citecolor=green]{hyperref}

\voffset=-2cm
\hoffset=-3cm
\textwidth = 19.5cm
\textheight= 23 cm

\usepackage{iwona}
\usepackage{fancyhdr}
\pagestyle{fancy}
\fancyhf{}
\fancyhead[RE,LO]{\bfseries{Seminario de Topología B}}
\fancyhead[LE,RO]{\bfseries{2019-2}}
\fancyfoot[RE,RO]{\bfseries{Marzo 2019}}
\fancyfoot[LE,LO]{\bfseries{Tarea 02}}

\newcommand{\R}{\mathbb R}
\newcommand{\Q}{\mathbb Q}
\newcommand{\E}{\mathbb E}
\newcommand{\s}{\mathbb S}
\newcommand{\C}{\mathbb C}
\newcommand{\F}{\mathbb F}
\newcommand{\T}{\mathbb T}
\newcommand{\p}{\mathbb P}
\newcommand{\I}{\mathbb I}
\newcommand{\A}{\mathbb A}


\begin{document}
\begin{center}
\textcolor{blue}{\textbf{\large Guía de ejercicios para al Evaluación Parcial 02}}\\
\vspace{0.5 cm}
\textcolor{red}{\textbf{\large EXAMEN PARCIAL 02 \\ VIERNES
29-MARZO-2019\\ De 19:00 a 21:00 HORAS - Salón P-108}}
\end{center}

\begin{enumerate}

\item Sean $(X,\tau_X)$, $(Y,\tau_Y)$ y $(Z, \tau_Z)$ espacios topológicos. Consideremos $f: X \to Y$, $g: X \to Y$, $\varphi: Y \to Z$ y $\psi: Y \to Z$ funciones continuas. Demostrar que:
\begin{enumerate}
\item Si $f \simeq g$ relativo a $A \subseteq X$ entonces $\varphi \circ f \simeq \varphi\circ g$ relativo a $A$.
\item Si $\varphi \simeq \phi$ relativo a $B\subseteq Y$ entonces $\varphi \circ f \simeq \psi \circ f$ relativo a $f^{-1}[B]$
\end{enumerate}

\item Sean $(X,\tau_X)$ y $(Y,\tau_Y)$ espacios topológicos. Demostrar que si $Y$ es conectable por trayectorias entonces para cualesquiera $f: X \to Y$ y $g : X\to Y$ funciones continuas nulhomotópicas se tiene que $f\simeq g$.

\item Demostrar que la relación de homotopía entre espacios topológicos es una relación de equivalencia en la clase de todos los espacios topológicos.

\item Sean $(X,\tau_X)$, $(Y,\tau_Y)$ y $(Z, \tau_Z)$ espacios topológicos.
\begin{enumerate}
\item Demostrar que si $\varphi: (X,x) \to (Y,y)$ y $\psi: (Y,y) \to (Z,z)$ son funciones basadas entonces
$$(\psi \circ \varphi)_* = \psi_*\circ \varphi_* : \pi(X,x)\to \pi(Z,z)$$

\item Demostrar que si $Id_X: (X,x)\to (X,x)$ es la función identidad en $X$.
$$(Id_X)_* = Id_{\pi(X,x)}$$
\end{enumerate}

\item Demostrar que si $(X, \tau_X)$ es un espacio topológico, $f : X \to \s^2$ y $g :X \to \s^2$ son funciones continuas tales que $\forall x\in X,\; f(x)\neq g(x)$ entonces $f\simeq g$.

\item Sean $(X, \tau_X)$ un espacio topológico simplemente conexo y $\{x_0,x_1\} \subseteq X$. Demostrar que si $f: I \to X$ y $g: I \to X$ son dos trayectorias con $f(0)=g(0)=x_0$ y $f(1)=g(1)=x_1$ entonces $f\simeq g$.

\item Sean $(Y,\tau_Y)$ y $(Z, \tau_Z)$ espacios topológicos y $\rho: Y \to Z$ una función cubriente. Demostrar que para todo $z\in Z$ el subespacio $p^{-1}(z)$ es un espacio discreto en $Y$.

\item Encontrar una función cubriente de $\s^1$ en $\s^1$ distinta de la función identidad.

\item Demostrar que $\R$ es un espacio cubriente de $\s^1$.

\item Demostrar que $\R^2$ es un espacio cubriente de la botella de Klein.

\end{enumerate}

\end{document}
