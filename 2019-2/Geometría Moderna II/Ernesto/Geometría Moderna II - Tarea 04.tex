\documentclass[11pt]{report}

\usepackage{amssymb}
\usepackage{amsmath}
\usepackage{amscd}
\usepackage{amsthm}
\usepackage[utf8]{inputenc}
\usepackage[spanish,mexico]{babel}
\usepackage{enumerate}
\usepackage[usenames]{color}
\numberwithin{section}{chapter}

\usepackage{pgf,tikz}
\usetikzlibrary{arrows}

\usepackage[colorlinks=true, linkcolor=blue, urlcolor=red,
citecolor=green]{hyperref}

\voffset=-2cm
\hoffset=-3.75cm
\textwidth = 20cm
\textheight= 23 cm

\usepackage{iwona}
\usepackage{fancyhdr}
\pagestyle{fancy}
\fancyhf{}
\fancyhead[RE,LO]{\bfseries{Geometría Moderna II}}
\fancyhead[LE,RO]{\bfseries{2019-2}}
\fancyfoot[RE,RO]{\bfseries{Mayo 2019}}
\fancyfoot[LE,LO]{\bfseries{Tarea 04}}

\newcommand{\R}{\mathbb R}
\newcommand{\Q}{\mathbb Q}
\newcommand{\E}{\mathbb E}
\newcommand{\s}{\mathbb S}
\newcommand{\C}{\mathbb C}
\newcommand{\h}{\mathbb H}
\newcommand{\F}{\mathbb F}
\newcommand{\T}{\mathbb T}
\newcommand{\p}{\mathbb P}
\newcommand{\I}{\mathbb I}
\newcommand{\A}{\mathbb A}


\begin{document}
\begin{center}
\textcolor{blue}{\textbf{\large Guía de ejercicios para al Evaluación Parcial 04}}\\
\vspace{0.5 cm}
\textcolor{red}{\textbf{\large EVALUACIÓN PARCIAL 03 \\ VIERNES 17-MAYO-2019\\ 20:00 HORAS - Salón P-108}}
\end{center}

\textbf{INSTRUCCIONES}:
\begin{itemize}
\item La cuarta evaluación parcial consistirá en una Tarea-Examen.
\item Cada estudiante deberá entregar por escrito únicamente cinco ejercicios de la siguiente lista con la misma paridad. De entregar más ejercicios, se anularán los ejercicios de mayor puntaje.
\end{itemize}

\begin{center}
\textcolor{purple}{\textbf{\large Geometría hiperbólica}}
\end{center}

\begin{center}
\textcolor{brown}{\textbf{\large Modelos del disco de Poincaré}}
\end{center}

\begin{enumerate}
\item Demostrar que dados dos puntos distintos en $\Delta$ existe una única recta hiperbólica en la que inciden.

\item Demostrar que dado un punto un segmento en $\Delta$ existe una circunferencia hiperbólica con centro el punto dado y radio la longitud del segmento dado.

\item Demostrar que dado un punto y una recta no incidentes en $\Delta$ existen rectas incidentes en el punto dado que no inciden en la recta dada.


\begin{center}
\textcolor{brown}{\textbf{\large Modelos del Semiplano}}
\end{center}

\item Demostrar que dados dos puntos distintos en $\h^2$ existe una única recta hiperbólica en la que inciden.

\item Demostrar que dado un punto un segmento en $\h^2$ existe una circunferencia hiperbólica con centro el punto dado y radio la longitud del segmento dado.

\item Demostrar que dado un punto y una recta no incidentes en $\h^2$ existen rectas incidentes en el punto dado que no inciden en la recta dada.


\begin{center}
\textcolor{brown}{\textbf{\large Cualquier modelo}}
\end{center}


\item Dados $\{A,B\} \subseteq \h^2$ puntos distintos, construir el punto medio del $h-$segmento $AB$.

\item Dado $P \in \h^2$ y $l$ una $h$-recta, construir una $h$-recta $m$ tal que $P \in m$ y $m$ sea ortogonal a $l$. ¿Es única dicha $h$-recta $m$?

%\item Demostrar que el lugar geométrico de los puntos en el plano cuya distancia a dos puntos fijos es la misma, es la $h-$recta ortogonal al $h-$segmento que determinan los puntos fijos por el punto medio del $h-$segmento.

\item Dadas dos $h$-rectas $l$ y $m$ tales que $l \cap m = \emptyset$, construir una $h$-recta $n$ que sea ortogonal a ambas. ¿Es única dicha $h$-recta $n$?

\item Dada una $e$-circunferencia totalmente contenida en $\h^2$, Determinar el centro hiperbólico de la  $h$-circunfeencia.

%\item  Sean $A, B, O$ tres puntos en el plano hiperbólico. Construir la $h-$circunferencia con $h-$centro $O$ y radio la distancia hiperbólica entre $A$ y $B$.

\item Sea $l$ una $h$-recta, ¿cuál es el lugar geométrico de los puntos $X$ en el plano hiperbólico cuya distancia hiperbólica a $l$ es la misma?

\item Sean $l$ y $m$  dos $h$-rectas incidentes en el plano hiperbólico. ¿Cuál es el lugar geométrico de los puntos $X$ en el plano hiperbólico tales que la distancia hiperbólica entre $X$ y $l$ es la misma que la distancia hiperbólica entre $X$ y $m$? 

%\item Demostrar que las medianas de un triángulo hiperbólico son concurrentes.

%\item Demostrar que las bisectrices de un triángulo hiperbólico son concurrentes.

\end{enumerate}

\end{document}
