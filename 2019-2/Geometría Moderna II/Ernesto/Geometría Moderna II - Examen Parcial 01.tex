\documentclass[12pt]{report}

\usepackage{amssymb,amsmath}
\usepackage[utf8]{inputenc}
\usepackage[spanish,mexico]{babel}

\usepackage{iwona}

\voffset=-3cm
\hoffset=-3cm
\textwidth=20 cm
\textheight=26 cm

\newcommand{\R}{\mathbb R}
\newcommand{\Q}{\mathbb Q}
\newcommand{\E}{\mathbb E}
\newcommand{\s}{\mathbb S}
\newcommand{\C}{\mathbb C}
\newcommand{\F}{\mathbb F}
\newcommand{\T}{\mathbb T}
\newcommand{\p}{\mathbb P}
\newcommand{\I}{\mathbb I}
\newcommand{\A}{\mathbb A}

\begin{document}

\begin{center}
\textbf{\LARGE {GEOMETRÍA MODERNA II}}
\end{center}

\begin{center}
\textbf{{\large 2019-2 (08 marzo 2019)}}
\end{center}

\begin{center}
\textbf{{\large EXAMEN PARCIAL 01}}
\end{center}

{\bf INSTRUCCIONES:} Justificar y argumentar todos los resultados que se realicen. Resolver únicamente cuatro ejercicios, de entregar más de cuatro ejercicios se anulará el ejercicio de mayor puntaje.

\begin{enumerate}

%\item Sea $O$ un punto en el plano, $r \in \R^+$ y consideremos a $\zeta (O,r)$. Demostrar que si $l$ y $m$ dos rectas en el plano tales que $l \cap m = \{O\}$; $l \cap \zeta(O,r) = \{A,B\}$ y $m \cap \zeta(O,r) = \{C,D\}$ entonces cualquier par de conjugados armónicos de $A$ y $B$ y cualquier par de conjugados armónicos de $C$ y $D$ pertenecen a una misma circunferencia.

\item Sea $\triangle ABC$; $h_X$ la altura por $X \in \{A,B,C\}$. Demostrar que si $h_A \cap \overline{BC}=\{D\}$, $h_B \cap \overline{CA}=\{E\}$ y $h_c \cap \overline{AB}=\{F\}$ entonces $D\{\overline{DE},\overline{DF};\overline{DA},\overline{DB}\} = -1$.

%\item Sean $\zeta(A, \alpha)$, $\zeta(B, \beta)$ circunferencias con $A \neq B$. Construir el eje radical de $\zeta(A, \alpha)$ y $\zeta(B, \beta)$ sin usar de $A$, $B$ y $\overline{AB}$.

\item Sea $\Gamma$ una familia de circunferencias coaxiales. Demostrar que para cualquier punto $P$ en el plano existe $\zeta(A, \alpha) \in \Gamma$ y $\zeta(B,\beta) \in \Gamma^\bot$ tal que $P \in \zeta(A, \alpha) \cap \zeta(B,\beta)$.
 

\item Sean $\zeta(A,\alpha)$ y $\zeta(B,\beta)$ circunferencias con $A \neq B$ cuyo eje radical es la recta $l$. Demostrar que para cualquier $C \in l$ si $m$ y $n$ son rectas tales que $m \cap n = \{C\}$ y $m \cap \zeta(A, \alpha) = \{P,Q\}$ y $n \cap \zeta(B, \beta) = \{R,S\}$ entonces $\{P,Q,R,S\}$ es un conjunto concíclico de puntos.

\item Demostrar que si $\{A,B,C\}$ es un conjunto de puntos en posición general y $\{\alpha, \beta,\gamma\} \subseteq \R$ entonces existe una circunferencia ortogonal a $\zeta(A,\alpha)$, $\zeta(B,\beta)$ y $\zeta(C,\gamma)$ simultáneamente.

\item Sea $\Gamma$ una familia de circunferencias coaxiales y $\zeta(A, \alpha) \notin \Gamma$. Demostrar que los ejes radicales de $\zeta (A, \alpha)$ y cada circunferencia de $\Gamma$ son rectas concurrentes.

%\item Sean $\{A,B,C\}$ un conjunto de puntos en posición general, $\{\alpha, \beta, \gamma\}\subseteq \R$ tales que $\zeta(A, \alpha)$, $\zeta(B, \beta)$ y $\zeta(C, \gamma)$ son circunferencias que no se intersecan por pares. Demostrar que los puntos seis puntos límite de las tres familias de circunferencias coaxiales que determinan por pares forman un conjunto concíclico de puntos.


\end{enumerate}

\vspace{2cm}


\begin{center}
\textbf{\LARGE {GEOMETRÍA MODERNA II}}
\end{center}

\begin{center}
\textbf{{\large 2019-2 (08 marzo 2019)}}
\end{center}

\begin{center}
\textbf{{\large EXAMEN PARCIAL 01}}
\end{center}

{\bf INSTRUCCIONES:} Justificar y argumentar todos los resultados que se realicen. Resolver únicamente cuatro ejercicios, de entregar más de cuatro ejercicios se anulará el ejercicio de mayor puntaje.

\begin{enumerate}

%\item Sea $O$ un punto en el plano, $r \in \R^+$ y consideremos a $\zeta (O,r)$. Demostrar que si $l$ y $m$ dos rectas en el plano tales que $l \cap m = \{O\}$; $l \cap \zeta(O,r) = \{A,B\}$ y $m \cap \zeta(O,r) = \{C,D\}$ entonces cualquier par de conjugados armónicos de $A$ y $B$ y cualquier par de conjugados armónicos de $C$ y $D$ pertenecen a una misma circunferencia.

\item Sea $\triangle ABC$; $h_X$ la altura por $X \in \{A,B,C\}$. Demostrar que si $h_A \cap \overline{BC}=\{D\}$, $h_B \cap \overline{CA}=\{E\}$ y $h_c \cap \overline{AB}=\{F\}$ entonces $D\{\overline{DE},\overline{DF};\overline{DA},\overline{DB}\} = -1$.

%\item Sean $\zeta(A, \alpha)$, $\zeta(B, \beta)$ circunferencias con $A \neq B$. Construir el eje radical de $\zeta(A, \alpha)$ y $\zeta(B, \beta)$ sin usar de $A$, $B$ y $\overline{AB}$.

\item Sea $\Gamma$ una familia de circunferencias coaxiales. Demostrar que para cualquier punto $P$ en el plano existe $\zeta(A, \alpha) \in \Gamma$ y $\zeta(B,\beta) \in \Gamma^\bot$ tal que $P \in \zeta(A, \alpha) \cap \zeta(B,\beta)$.
 

\item Sean $\zeta(A,\alpha)$ y $\zeta(B,\beta)$ circunferencias con $A \neq B$ cuyo eje radical es la recta $l$. Demostrar que para cualquier $C \in l$ si $m$ y $n$ son rectas tales que $m \cap n = \{C\}$ y $m \cap \zeta(A, \alpha) = \{P,Q\}$ y $n \cap \zeta(B, \beta) = \{R,S\}$ entonces $\{P,Q,R,S\}$ es un conjunto concíclico de puntos.

\item Demostrar que si $\{A,B,C\}$ es un conjunto de puntos en posición general y $\{\alpha, \beta,\gamma\} \subseteq \R$ entonces existe una circunferencia ortogonal a $\zeta(A,\alpha)$, $\zeta(B,\beta)$ y $\zeta(C,\gamma)$ simultáneamente.

\item Sea $\Gamma$ una familia de circunferencias coaxiales y $\zeta(A, \alpha) \notin \Gamma$. Demostrar que los ejes radicales de $\zeta (A, \alpha)$ y cada circunferencia de $\Gamma$ son rectas concurrentes.

%\item Sean $\{A,B,C\}$ un conjunto de puntos en posición general, $\{\alpha, \beta, \gamma\}\subseteq \R$ tales que $\zeta(A, \alpha)$, $\zeta(B, \beta)$ y $\zeta(C, \gamma)$ son circunferencias que no se intersecan por pares. Demostrar que los puntos seis puntos límite de las tres familias de circunferencias coaxiales que determinan por pares forman un conjunto concíclico de puntos.


\end{enumerate}



\end{document}
