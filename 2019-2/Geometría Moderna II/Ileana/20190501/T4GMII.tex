\documentclass[12pt]{article}

\usepackage[T1]{fontenc}
\usepackage{lmodern}

\usepackage{amssymb}
\usepackage{amsmath}
\usepackage{amscd}
\usepackage{amsthm}
\usepackage[utf8]{inputenc}
\usepackage[spanish,mexico]{babel}
\usepackage{enumerate}
\usepackage{pgf,tikz,pgfplots}
\usepackage{makeidx}
\usetikzlibrary{arrows}
\usepackage{multicol}
\usepackage{mathrsfs}

\usepackage[colorlinks=true, linkcolor=blue, urlcolor=blue, citecolor=red]{hyperref}

\usepackage{graphicx}
\usepackage{wrapfig}

\voffset=-2 cm
\hoffset=-3 cm
\textwidth=20 cm
\textheight=22 cm


\theoremstyle{definition}
\newtheorem{teo}{\textcolor{red}{Teorema}}[section]
\newtheorem{df}[teo]{\textcolor{cyan}{Definición}}
\newtheorem{ej}[teo]{\textcolor{red}{Ejercicio}}
\newtheorem{prop}[teo]{\textcolor{red}{Proposición}}
\newtheorem{lema}[teo]{\textcolor{red}{Lema}}
\newtheorem{cor}[teo]{\textcolor{red}{Corolario}}
\newtheorem{obs}[teo]{\textcolor{purple}{Observaci?n}}
\newtheorem{ejp}[teo]{\textcolor{purple}{Ejemplo}}

\newenvironment{pba}{\noindent\textbf{\textcolor{blue}{Prueba:}}}{\begin{flushright}
$\square$ \end{flushright}}
\newenvironment{dem}{\noindent\textbf{\textcolor{blue}{Demostraci?n}:}}{\begin{flushright}
\rule{1ex}{1ex} \end{flushright}}

\newcommand{\N}{\mathbb N}
\newcommand{\Z}{\mathbb Z}
\newcommand{\Q}{\mathbb Q}
\newcommand{\R}{\mathbb R}


\title{Tarea 4 \\
Geometría Moderna II}

\begin{document}

\maketitle

\begin{enumerate}

\item Sea $l_H$ y $P$ un punto en el plano hiperbólico. Construir una $m_H$ incidente en $P$ que sea ortogonal a $l_H$. ¿Es única dicha $h$-recta?

\item Sean $A$ y $B$ dos puntos en el plano hiperbólico. Construir el punto medio del $h-$segmento $AB$.

\item Demostrar que el lugar geométrico de los puntos en el plano cuya distancia a dos puntos fijos es la misma, es la $h-$recta ortogonal al $h-$segmento que determinan los puntos fijos por el punto medio del $h-$segmento.

\item Sean $l_H$ y $m_H$ no incidentes en el plano hiperbólico. Construir una $h-$recta perpendicular común a las dos. ¿Es única dicha $h$-recta?

\item Sea $h-\zeta$ una $h$-circunferencia en el plano hiperbólico. Encontrar el centro hiperbólico de $h-\zeta$.

\item  Sean $A, B, O$ tres puntos en el plano hiperbólico. Construir la $h-$circunferencia con $h-$centro $O$ y radio la distancia hiperbólica entre $A$ y $B$.

\item Sea $l_H$. ¿Cuál es el lugar geométrico de los puntos $X$ en el plano hiperbólico cuya distancia hiperbólica a $l_H$ es la misma?

\item Sean $l_H$ y $m_H$ incidentes en el plano hiperbólico. ¿Cuál es el lugar geométrico de los puntos $X$ en el plano hiperbólico tales que la distancia hiperbólico entre $X$ y $l_H$ es la misma que la distancia hiperbólica entre $X$ y $m_H$? 

\item Demostrar que las medianas de un triángulo hiperbólico son concurrentes.

\item Demostrar que las bisectrices de un triángulo hiperbólico son concurrentes.

\end{enumerate}
\end{document}