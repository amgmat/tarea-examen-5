\documentclass[10pt]{report}

\usepackage{amssymb,amsmath}
\usepackage[latin1]{inputenc}
\usepackage[spanish,mexico]{babel}

\usepackage{iwona}

\voffset=-3cm
\hoffset=-3.5 cm
\textwidth=19 cm
\textheight=30 cm

\begin{document}

\begin{center}
\textbf{\LARGE {GEOMETR�A MODERNA I}}
\end{center}

\begin{center}
\textbf{{\large 2019-1 (21 noviembre 2018)}}
\end{center}

\begin{center}
\textbf{{\large EXAMEN PARCIAL 04}}
\end{center}

{\bf INSTRUCCIONES:} Justificar y argumentar todos los resultados que se realicen. Resolver �nicamente cuatro ejercicios, de entregar m�s de cuatro ejercicios se anular� el ejercicio de mayor puntaje.

\begin{enumerate}

\item Sean $l$ y $m$ un par de rectas paralelas distintas. Encontrar el punto medio de $A$ y $B$ para cualquier $\{A,B\}\subseteq l$ con el uso de solamente regla.

\item Demostrar que dado $\triangle ABC$ si $P\in \overline{BC}$, $Q\in\overline{CA}$, $R\in\overline{AB}$ tales que $\overline{AP}\cap\overline{BQ}\cap\overline{CR}=\{O\}$ y $\overline{QR}\cap\overline{BC}=\{D\}$, $\overline{RP}\cap\overline{CA}=\{E\}$ y $\overline{PQ}\cap\overline{AB}=\{F\}$ entonces $D$, $E$ y $F$ son colineales. 

\item Demostrar que si $\zeta(I,r)$ es la circunferencia inscrita al $\triangle ABC$ y $\zeta(I,r) \cap\overline{BC}=\{P\}$, $\zeta(I,r)\cap\overline{CA}=\{Q\}$ y \break $\zeta(I,r)\cap\overline{AB}=\{R\}$ entonces $\overline{AP}$, $\overline{BQ}$ y $\overline{CR}$ son rectas concurrentes.

\item Sean $\triangle ABC$ y $\zeta (R,r)$ una circunferencia tal que $\zeta (R,r)\cap\overline{BC}=\{P,P'\}$,  $\zeta (R,r) \cap \overline{CA}=\{Q,Q'\}$, $ \zeta (R,r) \cap \overline{AB}=\{R,R'\}$. Si $\overline{AP}\cap \overline{BQ} \cap \overline{CR}\neq \emptyset$ entonces $\overline{AP'}\cap \overline{BQ'} \cap \overline{CR'}\neq \emptyset$

\item Sea $\zeta(O,r)$ una circunferencia y $\{A,B,C,D,E,F\} \subseteq \zeta(O,r)$ ordenados sobre ella (lev�giramente o dextr�giramente). Demostrar que la intersecci�n de los lados opuestos de hex�gono $ABCDEF$ inscrito en $\zeta(O,r)$ son tres puntos colineales.

\textbf{Sugerencia}: Considerar a $\overline{AB}\cap\overline{CD}=\{P\}$, $\overline{CD}\cap\overline{EF}=\{Q\}$ y $\overline{EF}\cap\overline{AB}=\{R\}$.

\item Encontrar la recta que contiene a un punto $P$ del plano y al punto de intersecci�n de dos rectas dadas sin tener acceso al punto de intersecci�n con el uso de solamente regla. 

\end{enumerate}

\vspace{3cm}

\begin{center}
\textbf{\LARGE {GEOMETR�A MODERNA I}}
\end{center}

\begin{center}
\textbf{{\large 2019-1 (21 noviembre 2018)}}
\end{center}

\begin{center}
\textbf{{\large EXAMEN PARCIAL 04}}
\end{center}

{\bf INSTRUCCIONES:} Justificar y argumentar todos los resultados que se realicen. Resolver �nicamente cuatro ejercicios, de entregar m�s de cuatro ejercicios se anular� el ejercicio de mayor puntaje.

\begin{enumerate}

\item Sean $l$ y $m$ un par de rectas paralelas distintas. Encontrar el punto medio de $A$ y $B$ para cualquier $\{A,B\}\subseteq l$ con el uso de solamente regla.

\item Demostrar que dado $\triangle ABC$ si $P\in \overline{BC}$, $Q\in\overline{CA}$, $R\in\overline{AB}$ tales que $\overline{AP}\cap\overline{BQ}\cap\overline{CR}=\{O\}$ y $\overline{QR}\cap\overline{BC}=\{D\}$, $\overline{RP}\cap\overline{CA}=\{E\}$ y $\overline{PQ}\cap\overline{AB}=\{F\}$ entonces $D$, $E$ y $F$ son colineales. 

\item Demostrar que si $\zeta(I,r)$ es la circunferencia inscrita al $\triangle ABC$ y $\zeta(I,r) \cap\overline{BC}=\{P\}$, $\zeta(I,r)\cap\overline{CA}=\{Q\}$ y \break $\zeta(I,r)\cap\overline{AB}=\{R\}$ entonces $\overline{AP}$, $\overline{BQ}$ y $\overline{CR}$ son rectas concurrentes.

\item Sean $\triangle ABC$ y $\zeta (R,r)$ una circunferencia tal que $\zeta (R,r)\cap\overline{BC}=\{P,P'\}$,  $\zeta (R,r) \cap \overline{CA}=\{Q,Q'\}$, $ \zeta (R,r) \cap \overline{AB}=\{R,R'\}$. Si $\overline{AP}\cap \overline{BQ} \cap \overline{CR}\neq \emptyset$ entonces $\overline{AP'}\cap \overline{BQ'} \cap \overline{CR'}\neq \emptyset$

\item Sea $\zeta(O,r)$ una circunferencia y $\{A,B,C,D,E,F\} \subseteq \zeta(O,r)$ ordenados sobre ella (lev�giramente o dextr�giramente). Demostrar que la intersecci�n de los lados opuestos de hex�gono $ABCDEF$ inscrito en $\zeta(O,r)$ son tres puntos colineales.

\textbf{Sugerencia}: Considerar a $\overline{AB}\cap\overline{CD}=\{P\}$, $\overline{CD}\cap\overline{EF}=\{Q\}$ y $\overline{EF}\cap\overline{AB}=\{R\}$.

\item Encontrar la recta que contiene a un punto $P$ del plano y al punto de intersecci�n de dos rectas dadas sin tener acceso al punto de intersecci�n con el uso de solamente regla. 

\end{enumerate}

\end{document}
